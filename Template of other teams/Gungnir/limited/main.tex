%!TEX program = xelatex
\documentclass[landscape, oneside, a4paper, cs4size]{book}

\def\marginset#1#2{                      % 页边设置 \marginset{left}{top}
\setlength{\oddsidemargin}{#1}         % 左边(书内侧)装订预留空白距离
\iffalse                   % 如果考虑左侧(书内侧)的边注区则改为\iftrue
\reversemarginpar
\addtolength{\oddsidemargin}{\marginparsep}
\addtolength{\oddsidemargin}{\marginparwidth}
\fi
\setlength{\evensidemargin}{0mm}       % 置0
\iffalse                   % 如果考虑右侧(书外侧)的边注区则改为\iftrue
\addtolength{\evensidemargin}{\marginparsep}
\addtolength{\evensidemargin}{\marginparwidth}
\fi
% \paperwidth = h + \oddsidemargin+\textwidth+\evensidemargin + h
\setlength{\hoffset}{\paperwidth}
\addtolength{\hoffset}{-\oddsidemargin}
\addtolength{\hoffset}{-\textwidth}
\addtolength{\hoffset}{-\evensidemargin}
\setlength{\hoffset}{0.5\hoffset}
\addtolength{\hoffset}{-1in}           % h = \hoffset + 1in
%\setlength{\voffset}{-1in}             % 0 = \voffset + 1in
\setlength{\topmargin}{\paperheight}
\addtolength{\topmargin}{-\headheight}
\addtolength{\topmargin}{-\headsep}
\addtolength{\topmargin}{-\textheight}
\addtolength{\topmargin}{-\footskip}
\addtolength{\topmargin}{#2}           % 上边预留装订空白距离
\setlength{\topmargin}{0.5\topmargin}
}
% 调整页边空白使内容居中,两参数分别为纸的左边和上边预留装订空白距离
\marginset{125mm}{200mm}


%\usepackage{ctex}
\usepackage{bm}
%\usepackage[fleqn]{amsmath}
\usepackage{harpoon}
\usepackage{fontspec}
\usepackage{listings}
\usepackage[left=1cm,right=1cm,top=1cm,bottom=1cm,columnsep=1cm,dvipdfm]{geometry}
\usepackage{setspace}
\usepackage{bm}
\usepackage{cmap}
\usepackage{cite}
\usepackage{float}
\usepackage{xeCJK}
\usepackage{amsthm}
\usepackage{amsmath}
\usepackage{amssymb}
\usepackage{multirow}
\usepackage{multicol}
\usepackage{setspace}
\usepackage{enumerate}
\usepackage{indentfirst}
\usepackage{adjmulticol}
\usepackage{titlesec}
\usepackage{color,minted}
\usepackage{xeCJK}
\allowdisplaybreaks
%\setlength{\parindent}{0em}
%\setlength{\mathindent}{0pt}
\lstset{breaklines}
\let\cleardoublepage\relax
\titleformat{\chapter}{\normalfont\large\bfseries}{Chapter \,\thechapter}{10pt}{\large}
\titleformat{\section}{\normalfont\normalsize\bfseries}{\thesection}{1em}{}
\titleformat{\subsection}{\normalfont\small\bfseries}{\thesubsection}{1em}{}
\titleformat{\subsubsection}{\normalfont\footnotesize\bfseries}{\thesubsubsection}{1em}{}
\titlespacing*{\chapter} {0pt}{0pt}{0pt}
\titlespacing*{\section} {0pt}{0pt}{0pt}
\titlespacing*{\subsection} {0pt}{-1pt}{-1pt}
\titlespacing*{\subsubsection}{0pt}{-1pt}{-1pt}
%configure fonts
\setmonofont{Menlo}[Scale=0.8]
\setCJKmainfont{STHeiti}
\usepackage{yfonts}

\renewcommand{\theFancyVerbLine}{\sffamily \textcolor[rgb]{0.5,0.5,0.5}{\scriptsize {\arabic{FancyVerbLine}}}}

\setminted[cpp]{
	style=xcode,
	mathescape,
	linenos,
	autogobble,
	baselinestretch=0.5,
	tabsize=4,
	fontsize=\small,
	%bgcolor=Gray,
	frame=single,
	framesep=1mm,
	framerule=0.3pt,
	numbersep=1mm,
	breaklines=true,
	breaksymbolsepleft=2pt,
	%breaksymbolleft=\raisebox{0.8ex}{ \small\reflectbox{\carriagereturn}}, %not moe!
	%breaksymbolright=\small\carriagereturn,
	breakbytoken=false,
}
\setminted[java]{
	style=xcode,
	mathescape,
	linenos,
	autogobble,
	baselinestretch=1.0,
	tabsize=4,
	%bgcolor=Gray,
	frame=single,
	framesep=1mm,
	framerule=0.3pt,
	numbersep=1mm,
	breaklines=true,
	breaksymbolsepleft=2pt,
	%breaksymbolleft=\raisebox{0.8ex}{ \small\reflectbox{\carriagereturn}}, %not moe!
	%breaksymbolright=\small\carriagereturn,
	breakbytoken=false,
}
\setminted[text]{
	style=xcode,
	mathescape,
	linenos,
	autogobble,
	baselinestretch=1.0,
	tabsize=4,
	%bgcolor=Gray,
	frame=single,
	framesep=1mm,
	framerule=0.3pt,
	numbersep=1mm,
	breaklines=true,
	breaksymbolsepleft=2pt,
	%breaksymbolleft=\raisebox{0.8ex}{ \small\reflectbox{\carriagereturn}}, %not moe!
	%breaksymbolright=\small\carriagereturn,
	breakbytoken=false,
}
\begin{document}\scriptsize
	\renewcommand{\thefootnote}{\fnsymbol{footnote}}
	\title{\Huge{\textgoth{Gungnir's Standard Code Library}}\thanks{https://github.com/footoredo/Gungnir}}
	\author{\emph{Shanghai Jiao Tong University}}
	\date{Dated: \today}
	\maketitle
	\clearpage
	\begin{multicols}{2}
		\tableofcontents
		\clearpage
		\begin{spacing}{0.8}
			\def \source {../source}
\chapter{数论}
\section{$O(m^2\log n)$线性递推}
Given $a_0, a_1, \ldots, a_{m - 1}$\\
	$a_n = c_0 \times a_{n - m} + \cdots + c_{m - 1} \times a_{n - 1}$\\
	Solve for $a_n = v_0 \times a_0 + v_1 \times a_1 + \cdots + v_{m - 1} \times a_{m - 1}$\\
\inputminted{cpp}{\source/number-theory/linear-recurrence.cpp}
\section{求逆元}
\inputminted{cpp}{\source/number-theory/get-inversion.cpp}
\section{中国剩余定理}
\inputminted{cpp}{\source/number-theory/chinese-remainder-theorem.cpp}
\section{素性测试}
\inputminted{cpp}{\source/number-theory/primality-test.cpp}
\section{质因数分解}
\inputminted{cpp}{\source/number-theory/pollards-rho-algorithm.cpp}
\section{佩尔方程}
\inputminted{java}{\source/number-theory/Pell.java}
\section{二次剩余}
\inputminted{cpp}{\source/number-theory/square.cpp}
\section{一元三次方程}
\inputminted{cpp}{\source/number-theory/cubic-polynomial.cpp}
\section{线下整点}
\inputminted{cpp}{\source/number-theory/integer-lattice-under-segment.cpp}
\section{线性同余不等式}
\inputminted{cpp}{\source/number-theory/linear-inequaltion.cpp}
\section{组合数取模}
\inputminted{cpp}{\source/number-theory/module.cpp}
\section{Schreier-Sims}
\inputminted{cpp}{\source/number-theory/SchreierSims.cpp}
\section{分治FFT}
\inputminted{cpp}{\source/number-theory/cdq-fft.cpp}

\input{\source/nemurical-approach.tex}
\input{\source/geography.tex}
\chapter{字符串}
\section{AC自动机}
\inputminted{cpp}{\source/string/Aho-Corasick-automaton.cpp}
\section{后缀数组}
\inputminted{cpp}{\source/string/suffix-array.cpp}
\section{后缀自动机}
\inputminted{cpp}{\source/string/suffix-automaton.cpp}
\section{广义后缀自动机}
\inputminted{cpp}{\source/string/ex-suffix-automaton.cpp}
\section{manacher}
\inoutminted{cpp}{\source/string/Manacher.cpp}
\section{回文自动机}
\inputminted{cpp}{\source/string/palindromic-tree.cpp}
\section{循环串的最小表示}

\chapter{数据结构}
\section{可并堆}
\inputminted{cpp}{\source/data-structure/heap.cpp}
\section{KD-Tree}
\inputminted{cpp}{\source/data-structure/kd-tree.cpp}
\section{Treap}
\inputminted{cpp}{\source/data-structure/treap.cpp}
\section{Splay}
\inputminted{cpp}{\source/data-structure/spaly.cpp}
\section{Link cut Tree}
\inputminted{cpp}{\source/data-structure/link-cut-tree.cpp}
\section{树上莫队}
\inputminted{cpp}{\source/data-structure/mo'alogorithm-on-tree.cpp}
\section{CDQ分治}
\inputminted{cpp}{\source/data-structure/cdq.cpp}
\section{整体二分}
\inputminted{cpp}{\source/data-structure/general-binary.cpp}

\section{图论}
%\subsection{2-SAT}
%\inputminted{cpp}{\source/graph-theory/sat-lexicographically.cpp}
\subsection{2-SAT tarjan}
\inputminted{cpp}{\source/graph-theory/2-sat-tarjan.cpp}
\subsection{KM}
\inputminted{cpp}{\source/graph-theory/KM.cpp}
\subsection{点双连通分量}
\inputminted{cpp}{\source/graph-theory/biconnected-graph-vertex.cpp}
\subsection{边双连通分量}
\inputminted{cpp}{\source/graph-theory/biconnected-graph-edge.cpp}
\subsection{最小树形图}
\inputminted{cpp}{\source/graph-theory/optimum-branching.cpp}
\subsection{带花树}
\inputminted{cpp}{\source/graph-theory/blossom-algorithm.cpp}
\subsection{支配树}
\inputminted{cpp}{\source/graph-theory/dominator-tree.cpp}
\subsection{无向图最小割}
\inputminted{cpp}{\source/graph-theory/stoer-wagner-algorithm.cpp}
\subsection{最大团搜索}
\inputminted{cpp}{\source/graph-theory/maxclique.cpp}
%\subsection{弦图判定}
%\inputminted{cpp}{\source/graph-thery/judge.cpp}
%\subsection{斯坦纳树}
%\inputminted{cpp}{\source/graph-theory/Steiner-Tree.cpp}
\subsection{虚树}
\inputminted{cpp}{\source/graph-theory/mirage-tree.cpp}
\subsection{点分治}
\inputminted{cpp}{\source/graph-theory/vertex-partition.cpp}
\subsection{最小割最大流}
\inputminted{cpp}{\source/graph-theory/dinic.cpp}
\subsection{最小费用流}
\inputminted{cpp}{\source/graph-theory/mincost-maxflow.cpp}
%\subsection{zkw费用流}
%\inputminted{cpp}{\source/graph-theory/zkw-cost-flow.cpp}
\subsection{最小割树}
\inputminted{cpp}{\source/graph-theory/GH-tree.cpp}
\subsection{上下界网络流建图}
$B(u,v)$表示边$(u,v)$流量的下界,$C(u,v)$表示边$(u,v)$流量的上界,$F(u,v)$表示边$(u,v)$的流量。
设$G(u,v) = F(u,v) - B(u,v)$,显然有
$$0 \leq G(u,v) \leq C(u,v)-B(u,v)$$
\subsubsection{无源汇的上下界可行流}
建立超级源点$S^*$和超级汇点$T^*$,对于原图每条边$(u,v)$在新网络中连如下三条边:$S^* \rightarrow v$,容量为$B(u,v)$;$u \rightarrow T^*$,容量为$B(u,v)$;$u \rightarrow v$,容量为$C(u,v) - B(u,v)$。最后求新网络的最大流,判断从超级源点$S^*$出发的边是否都满流即可,边$(u,v)$的最终解中的实际流量为$G(u,v)+B(u,v)$。
\subsubsection{有源汇的上下界可行流}
从汇点$T$到源点$S$连一条上界为$\infty$,下界为$0$的边。按照\textbf{无源汇的上下界可行流}一样做即可,流量即为$T \rightarrow S$边上的流量。
\subsubsection{有源汇的上下界最大流}
\begin{enumerate}
	\item 在\textbf{有源汇的上下界可行流}中,从汇点$T$到源点$S$的边改为连一条上界为$\infty$,下届为$x$的边。$x$满足二分性质,找到最大的$x$使得新网络存在\textbf{无源汇的上下界可行流}即为原图的最大流。
	\item 从汇点$T$到源点$S$连一条上界为$\infty$,下界为$0$的边,变成无源汇的网络。按照\textbf{无源汇的上下界可行流}的方法,建立超级源点$S^*$和超级汇点$T^*$,求一遍$S^* \rightarrow T^*$的最大流,再将从汇点$T$到源点$S$的这条边拆掉,求一次$S \rightarrow T$的最大流即可。
\end{enumerate}
\subsubsection{有源汇的上下界最小流}
\begin{enumerate}
	\item 在\textbf{有源汇的上下界可行流}中,从汇点$T$到源点$S$的边改为连一条上界为$x$,下界为$0$的边。$x$满足二分性质,找到最小的$x$使得新网络存在\textbf{无源汇的上下界可行流}即为原图的最小流。
	\item 按照\textbf{无源汇的上下界可行流}的方法,建立超级源点$S^*$与超级汇点$T^*$,求一遍$S^* \rightarrow T^*$的最大流,但是注意这一次不加上汇点$T$到源点$S$的这条边,即不使之改为无源汇的网络去求解。求完后,再加上那条汇点$T$到源点$S$上界$\infty$的边。因为这条边下界为$0$,所以$S^*$,$T^*$无影响,再直接求一次$S^* \rightarrow T^*$的最大流。若超级源点$S^*$出发的边全部满流,则$T \rightarrow S$边上的流量即为原图的最小流,否则无解。
\end{enumerate}

\chapter{其他}
\section{Dancing Links}
\inputminted{cpp}{\source/others/dancing-links.cpp}
\section{蔡勒公式}
\inputminted{cpp}{\source/others/zellers-congruence.cpp}
%\section{五边形数定理}
%the number of partitions of n:
%$p(n) = \sum_{k \in \mathbb{Z} \backslash \{0\}} (-1)^{k - 1}p(n - \frac{k(3k-1)}{2})$

\chapter{技巧}
\sectio{STL归还空间}
\inputminted{cpp}{\source/tricks/truly-release-container-space.cpp}
\section{大整数取模}
\inputminted{cpp}{\source/tricks/O1-multiply-mod.cpp}
\section{读入优化}
\inputminted{cpp}{\source/tricks/unbeatable-input-acceleration.cpp}
\section{二次随机法}
\inputminted{cpp}{\source/tricks/mersenne-twister.cpp}
\section{vimrc}
\inputminted{cpp}
\section{控制cout输出实数精度}
\inputminted{cpp}{\source/tricks/control-cout-precision.cpp}
\section{让make支持c++11}
in .bashrc or whatever
\begin{verbatim}
export CXXFLAGS='-std=c++11 -Wall'
\end{verbatim}
\section{tuple相关}
\inputminted{cpp}{\source/tricks/tuple.cpp}

\section{提示}

\subsection{线性规划转对偶}

\begin{equation*}
\begin{aligned}
&\text{maximize }\mathbf{c}^{T}\mathbf{x}\\
&\text{subject to }\mathbf{A}\mathbf{x} \leq \mathbf{b}, \mathbf{x} \geq 0
\end{aligned}
\Longleftrightarrow
\begin{aligned}
&\text{minimize }\mathbf{y}^{T}\mathbf{b}\\
&\text{subject to }\mathbf{y}^{T}\mathbf{A} \geq \mathbf{c}^{T}, \mathbf{y} \geq 0
\end{aligned}
\end{equation*}

\subsection{NTT 素数及其原根}
\begin{tabular}{|l|l|}
\hline
\texttt{Prime} & \texttt{Primitive root} \\
\hline
1053818881 & 7 \\
\hline
1051721729 & 6 \\
\hline
1045430273 & 3 \\
\hline
1012924417 & 5 \\
\hline
1007681537 & 3 \\
\hline
\end{tabular}

\subsection{积分表}
\newcommand{\md}{\mathrm{d}}
\newcommand{\me}{\mathrm{e}}

\begin{small}

\subsubsection{$ax^2+bx+c$($a>0$)}

\begin{enumerate}

\item $ \int \frac{\md x}{ax^2+bx+c} = \begin{cases}
\frac{2}{\sqrt{4ac-b^2}}\arctan\frac{2ax+b}{\sqrt{4ac-b^2}} + C & (b^2 < 4ac) \\
\frac{1}{\sqrt{b^2-4ac}}\ln\left| \frac{2ax+b-\sqrt{b^2-4ac}}{2ax+b+\sqrt{b^2-4ac}} \right| + C & (b^2 > 4ac)
\end{cases} $

\item $ \int \frac{x}{ax^2+bx+c} \md x = \frac{1}{2a} \ln |ax^2+bx+c| - \frac{b}{2a} \int \frac{\md x}{ax^2+bx+c} $

\end{enumerate}

\subsubsection{$\sqrt{\pm ax^2+bx+c}$($a>0$)}

\begin{enumerate}

\item $ \int \frac{\md x}{\sqrt{ax^2+bx+c}} = \frac{1}{\sqrt{a}} \ln | 2ax+b+2\sqrt{a}\sqrt{ax^2+bx+c} | + C $

\item $ \int \sqrt{ax^2+bx+c} \md x = \frac{2ax+b}{4a}\sqrt{ax^2+bx+c} +
	\frac{4ac-b^2}{8\sqrt{a^3}}\ln |2ax+b+2\sqrt{a}\sqrt{ax^2+bx+c}| + C $

\item $ \int \frac{x}{\sqrt{ax^2+bx+c}} \md x = \frac{1}{a}\sqrt{ax^2+bx+c} -
	\frac{b}{2\sqrt{a^3}}\ln | 2ax+b+2\sqrt{a}\sqrt{ax^2+bx+c} | + C $

\item $ \int \frac{\md x}{\sqrt{c+bx-ax^2}} = -\frac{1}{\sqrt{a}} \arcsin \frac{2ax-b}{\sqrt{b^2+4ac}} + C  $

\item $ \int \sqrt{c+bx-ax^2} \md x = \frac{2ax-b}{4a}\sqrt{c+bx-ax^2} + \\
	\frac{b^2+4ac}{8\sqrt{a^3}}\arcsin\frac{2ax-b}{\sqrt{b^2+4ac}} + C $

\item $ \int \frac{x}{\sqrt{c+bx-ax^2}} \md x = -\frac{1}{a}\sqrt{c+bx-ax^2} + \frac{b}{2\sqrt{a^3}}\arcsin\frac{2ax-b}{\sqrt{b^2+4ac}} + C $

\end{enumerate}

\subsubsection{$\sqrt{\pm\frac{x-a}{x-b}}$或$\sqrt{(x-a)(x-b)}$}

\begin{enumerate}

\item $ \int \frac{\md x}{\sqrt{(x-a)(b-x)}} = 2\arcsin\sqrt\frac{x-a}{b-x} + C$ ($a<b$)

\item \begin{multline}
\int \sqrt{(x-a)(b-x)} \md x = \frac{2x-a-b}{4}\sqrt{(x-a)(b-x)} + \\
	\frac{(b-a)^2}{4}\arcsin\sqrt\frac{x-a}{b-x} + C, (a<b)
\end{multline}

\end{enumerate}

\subsubsection{三角函数的积分}

\begin{enumerate}

\item $ \int \tan x \md x = -\ln|\cos x| + C $

\item $ \int \cot x \md x = \ln |\sin x| + C $

\item $ \int \sec x \md x = \ln \left| \tan\left( \frac{\pi}{4} + \frac{x}{2} \right) \right| + C = \ln |\sec x + \tan x| + C $

\item $ \int \csc x \md x = \ln \left| \tan\frac{x}{2} \right| + C = \ln |\csc x - \cot x| + C $

\item $ \int \sec^2 x \md x = \tan x + C $

\item $ \int \csc^2 x \md x = -\cot x + C $

\item $ \int \sec x \tan x \md x = \sec x + C $

\item $ \int \csc x \cot x \md x = -\csc x + C $

\item $ \int \sin^2 x \md x = \frac{x}{2} - \frac{1}{4} \sin 2x + C $

\item $ \int \cos^2 x \md x = \frac{x}{2} + \frac{1}{4} \sin 2x + C $

\item $ \int \sin^n x \md x = -\frac{1}{n} \sin^{n-1} x \cos x + \frac{n-1}{n} \int \sin^{n-2} x \md x $

\item $ \int \cos^n x \md x = \frac{1}{n} \cos^{n-1} x \sin x + \frac{n-1}{n} \int \cos^{n-2} x \md x $

\item $ \int \frac{\md x}{\sin^n x} = -\frac{1}{n-1} \frac{\cos x}{\sin^{n-1}x} + \frac{n-2}{n-1} \int \frac{\md x}{\sin^{n-2}x} $

\item $ \int \frac{\md x}{\cos^n x} = \frac{1}{n-1} \frac{\sin x}{\cos^{n-1}x} + \frac{n-2}{n-1} \int \frac{\md x}{\cos^{n-2}x} $

\item \[ \begin{split} {} & \int \cos^m x \sin^n x \md x \\
	= & \frac{1}{m+n} \cos^{m-1} x \sin^{n+1}x + \frac{m-1}{m+n}\int\cos^{m-2}x\sin^nx\md x \\
	= & -\frac{1}{m+n} \cos^{m+1} x \sin^{n-1}x + \frac{n-1}{m+1} \int \cos^m x\sin^{n-2} x \md x \end{split} \]

\item $ \int \frac{\md x}{a + b \sin x} = \begin{cases}
\frac{2}{\sqrt{a^2-b^2}}\arctan\frac{a\tan\frac{x}{2}+b}{\sqrt{a^2-b^2}} + C & (a^2 > b^2) \\
\frac{1}{\sqrt{b^2-a^2}}\ln \left| \frac{a\tan\frac{x}{2}+b-\sqrt{b^2-a^2}}{a\tan\frac{x}{2}+b+\sqrt{b^2-a^2}} \right| + C & (a^2 < b^2)
\end{cases} $

\item $ \int \frac{\md x}{a + b \cos x} = \begin{cases}
\frac{2}{a+b}\sqrt\frac{a+b}{a-b} \arctan\left(\sqrt\frac{a-b}{a+b}\tan\frac{x}{2}\right) + C & (a^2 > b^2) \\
\frac{1}{a+b}\sqrt\frac{a+b}{a-b} \ln \left| \frac{\tan\frac{x}{2}+\sqrt\frac{a+b}{b-a}}{\tan\frac{x}{2}-\sqrt\frac{a+b}{b-a}} \right| + C
& (a^2 < b^2)
\end{cases} $

\item $ \int \frac{\md x}{a^2\cos^2x+b^2\sin^2x} = \frac{1}{ab} \arctan\left( \frac{b}{a}\tan x \right) + C $

\item $ \int \frac{\md x}{a^2\cos^2x-b^2\sin^2x} = \frac{1}{2ab}\ln\left|\frac{b\tan x+a}{b\tan x-a}\right| + C $

\item $ \int x \sin ax \md x = \frac{1}{a^2} \sin ax - \frac{1}{a} x \cos ax + C $

\item $ \int x^2 \sin ax \md x = -\frac{1}{a} x^2 \cos ax + \frac{2}{a^2} x \sin ax + \frac{2}{a^3} \cos ax + C$

\item $ \int x \cos ax \md x = \frac{1}{a^2} \cos ax + \frac{1}{a} x \sin ax + C $

\item $ \int x^2 \cos ax \md x = \frac{1}{a} x^2 \sin ax + \frac{2}{a^2} x \cos ax - \frac{2}{a^3} \sin ax + C $

\end{enumerate}

\subsubsection{反三角函数的积分(其中 $a>0$ )}

\begin {enumerate}

\item $ \int \arcsin \frac{x}{a} \md x = x \arcsin \frac{x}{a} + \sqrt{a^2-x^2}+C $

\item $ \int x \arcsin \frac{x}{a} \md x= (\frac{x^2}{2}-\frac{a^2}{4})\arcsin \frac{x}{a} + \frac{x}{4} \sqrt{x^2-x^2}+C$

\item $ \int x^2 \arcsin \frac{x}{a} \md x = \frac{x^3}{3}\arcsin \frac{x}{a}+\frac{1}{9}(x^2+2 a^2)\sqrt{a^2-x^2}+C $

\item $ \int \arccos \frac{x}{a} \md x= x \ arccos \frac{x}{a} - \sqrt{a^2-x^2} +C $

\item $ \int x \arccos \frac{x}{a} \md x= (\frac{x^2}{2}-\frac{a^2}{4})\arccos \frac{x}{a} - \frac{x}{4} \sqrt{a^2-x^2}+C $

\item $ \int x^2 \arccos \frac{x}{a}\md x= \frac{x^3}{3}\arccos \frac{x}{a} - \frac{1}{9}(x^2+2a^2)\sqrt{a^2-x^2}+C$

\item $ \int \arctan \frac{x}{a} \md x=x \arctan \frac{x}{a}-\frac{a}{2}\ln (a^2+x^2)+C $

\item $ \int x\arctan \frac{x}{a} \md x = \frac{1}{2}(a^2+x^2)\arctan \frac{x}{a} -\frac{a}{2}x+C $

\item $ \int x^2 \arctan \frac{x}{a} \md x= \frac{x^3}{3} \arctan \frac{x}{a} - \frac{a}{6}x^2 + \frac{a^3}{6} \ln (a^2+x^2)+C $

\end {enumerate}

\subsubsection{指数函数的积分}

\begin{enumerate}

\item $ \int a^x \md x= \frac{1}{\ln a} a^x + C$

\item $ \int \me ^{ax}\md x=\frac{1}{a}a^{ax}+C $

\item $ \int x \me  ^ {ax} \md x=\frac{1}{a^2}(ax-1)a^{ax} +C $

\item $ \int x^n \me ^{ax} \md x=\frac{1}{a}x^n \me ^{ax}-\frac{n}{a} \int x^{n-1} \me ^ {ax} \md x $

\item $ \int x a^x \md x = \frac{x}{\ln a}a^x-\frac{1}{(\ln a)^2}a^x+C $

\item $ \int x^n a^x \md x= \frac{1}{\ln a}x^n a^x-\frac{n}{\ln a}\int x^{n-1}a^x \md x $

\item $ \int \me ^{ax} \sin bx \md x = \frac{1}{a^2+b^2}\me ^{ax}(a \sin bx - b \cos bx)+C $

\item $ \int \me ^{ax} \cos bx \md x = \frac{1}{a^2+b^2}\me ^{ax}(b \sin bx + a \cos bx)+C $

\item $ \int \me ^{ax} \sin ^ n bx \md x=\frac{1}{a^2+b^2 n^2}\me ^{ax} \sin ^ {n-1} bx (a \sin bx -nb \cos bx) +\frac{n(n-1)b^2}{a^2+b^2 n^2}\int \me ^{ax} \sin ^{n-2} bx \md x $

\item $ \int \me ^{ax} \cos ^ n bx \md x=\frac{1}{a^2+b^2 n^2}\me ^{ax} \cos ^ {n-1} bx (a \cos bx +nb \sin bx) +\frac{n(n-1)b^2}{a^2+b^2 n^2}\int \me ^{ax} \cos ^{n-2} bx \md x $

\end{enumerate}

\subsubsection{对数函数的积分}

\begin{enumerate}

\item $ \int \ln x \md x = x \ln x - x + C$

\item $ \int \frac{\md x}{x \ln x} =\ln \big | \ln x \big |+C $

\item $ \int x^n \ln x \md x = \frac{1}{n+1}x^{n+1}(\ln x - \frac{1}{n+1} ) +C $

\item $ \int (\ln x)^{n} \md x = x(\ln x)^ n - n \int (\ln x)^{n-1} \md x $

\item $ \int x ^ m(\ln x)^n \md x=\frac{1}{m+1}x^{m+1} (\ln x)^n - \frac{n}{m+1} \int x^m(\ln x)^{n-1}\md x $

\end{enumerate}

\end{small}


		\end{spacing}
	\end{multicols}
\end{document}
