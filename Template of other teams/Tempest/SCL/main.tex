\documentclass[a4paper,10pt]{book}
\usepackage{amsmath}
\usepackage{amssymb}
\usepackage{fontspec}
\usepackage{listings} 
\usepackage{harpoon}
\usepackage[left=1.5cm, right=1.5cm]{geometry}
\usepackage[BoldFont]{xeCJK}
\oddsidemargin -0.1 true cm
\if@twoside
	\evensidemargin -0.1 true cm
\fi
\setlength{\parindent}{0em}
\setCJKmainfont{Microsoft YaHei}
\lstset{
	language=C++,
	numbers=left,
	tabsize=4,
	breaklines=tr,
	extendedchars=false
}

\title{\LARGE{Standard Code Library}}
\author{Tempest}
\date{October, 2014}
\begin{document}
\maketitle

\tableofcontents

\newpage

\chapter{}

\chapter{二维几何}
	\section{Naive Tips}
		\input{注意.tex}
	\section{几何公式}
		\input{几何公式.tex}
	\section{点类}
		\input{点类.tex}
	\section{基本操作}
		\input{基本操作.tex}
	\section{球面}
		\input{球面.tex}
	\section{半平面交}
		\input{半平面交.tex}
	\section{最小圆覆盖}
		\input{最小圆覆盖.tex}
	\section{判断圆存在交集$NlogK$}
		\input{判断圆存在交集.tex}
	\section{圆的交并面积}
		\input{SCL_cir_k_inter.tex}
	\section{Farmland}
		\subsection{Logic\_IU}
			\input{farmlandLYP.tex}
		\subsection{tEJtM}
			\input{SCL_farmland_GY.tex}

\chapter{三维几何}
	\section{点类$+$三维凸包$N^3+$凸包求重心}
		\input{Geo3.tex}
	\section{三维旋转}
		\input{三维旋转.tex}
	\section{最小球覆盖}
		随机增量法,复杂度没有找到靠谱证明,暂且可以类似最小圆覆盖当线性用
		\input{最小球覆盖.tex}
		
\chapter{图论}
    \section{Dijkstra}
	    求s到其他点的最短路
	    \input{dijkstra.tex}

	\section{最大流}
	    iSAP算法求S到T的最大流,点数为cntN,边表存储在*E[]中
	    \input{maxflow.tex}

	\section{上下界流}
		\subsection{上下界无源汇可行流}
			\input{上下界无源汇可行流.tex}
		\subsection{上下界最大流}
			\input{上下界最大流.tex}
		\subsection{上下界最小流}
			\input{上下界最小流.tex}
		\subsection{上下界有源汇可行流}
			\input{上下界有源汇可行流.tex}

	\section{费用流}
		\subsection{Logic\_IU$+$负费用路}
			注意图的初始化,费用和流的类型依题目而定
			\input{mincostflow.tex}
		\subsection{shytangyuan$+$ZKW}
			\input{zkw.tex}

	\section{强联通分量}
		\subsection{Logic\_IU}
			N个点的图求SCC,totID为时间标记,top为栈顶,totCol为强联通分量个数,注意初始化
			\input{SCC.tex}
		\subsection{shytangyuan$+$手写栈}
			\input{手工栈tarjan.tex}
	
	\section{KM}
		\subsection{tEJtM}
			\input{KM_GY.tex}
        \subsection{Logic\_IU}
	        求完备匹配的最大权匹配,建好的完全图用w[][]存储,点数为N
	        \begin{lstlisting}
#include <cstdio>
#include <cstdlib>
#include <algorithm>
#include <vector>
#include <cstring>
#include <string>
#include <iostream>

#define foreach(e, x) for(__typeof(x.begin()) e = x.begin(); e != x.end(); ++e)

using namespace std;

const int N = 333;
const int INF = (1 << 30);

int mat[N][N], lx[N], ly[N], vx[N], vy[N], slack[N];
int n, match[N];

bool find(int x) {
	vx[x] = 1;
	for(int i = 1; i <= n; i++) {
		if (vy[i]) {
			continue;
		}
		int temp = lx[x] + ly[i] - mat[x][i];
		if (temp == 0) {
			vy[i] = 1;
			if (match[i] == -1 || find(match[i])) {
				match[i] = x;
				return true;
			}
		} else {
			slack[i] = min(slack[i], temp);
		}
	}
	return false;
}

int KM() {
	for(int i = 1; i <= n; i++) {
		lx[i] = -INF;
		ly[i] = 0;
		match[i] = -1;
		for(int j = 1; j <= n; j++) {
			lx[i] = max(lx[i], mat[i][j]);
		}
	}
	for(int i = 1; i <= n; i++) {
		for(int j = 1; j <= n; j++) {
			slack[j] = INF;
		}
		for(; ;) {
			memset(vx, 0, sizeof(vx));
			memset(vy, 0, sizeof(vy));
			for(int j = 1; j <= n; j++) {
				slack[j] = INF;
			}
			if (find(i)) {
				break;
			}
			int delta = INF;
			for(int j = 1; j <= n; j++) {
				if (!vy[j]) {
					delta = min(delta, slack[j]);
				}
			}
			for(int j = 1; j <= n; j++) {
				if (vx[j]) {
					lx[j] -= delta;
				}
				if (vy[j]) {
					ly[j] += delta;
				} else {
					slack[j] -= delta;
				}
			}
		}
	}
	int answer = 0;
	for(int i = 1; i <= n; i++) {
		answer += mat[match[i]][i];
	}
	return answer;
}

int main() {
	while(scanf("%d", &n) != EOF) {
		for(int i = 1; i <= n; i++) {
			for(int j = 1; j <= n; j++) {
				scanf("%d", &mat[i][j]);
			}
		}
		printf("%d\n", KM());
	}
	return 0;
}
\end{lstlisting}

		\subsection{shytangyuan$+$邻接阵}
			\input{km_邻接矩阵.tex}
		\subsection{shytangyuan$+$链表}
			\input{km_链表.tex}
	\section{无向图最小割}
		\input{无向图最小割.tex}

\chapter{数据结构}
	\section{KD树}
		\subsection{tEJtM$+$高维}
			\input{KDTree_GY.tex}

	    \subsection{Logic\_IU}
	        读入N个点,输出距离每个点的最近点。
	        \input{KDtree.tex}

	\section{后缀自动机}
		\subsection{tEJtM$+$LCA非递归Tarjan}
			\input{SAM_GY.tex}
		\subsection{Logic\_IU}
			\input{suffix-automaton.tex}

    \section{Splay树}
		\subsection{Logic\_IU}
			注意初始化内存池和null节点,以及根据需要修改update和relax,区间必须是1-based
			\input{splay.tex}
		\subsection{shytangyuan}
			\input{pinhengshu.tex}

	\section{动态树}
		根据需求修改Node中的relax和update函数,修改access,以及Node的构造函数,注意初始化内存池和null节点
		\input{LCT.tex}

	\section{二叉堆}
		双射堆,ind[v]表示标号为v的节点在堆中的位置
		\input{binary-heap.tex}

	\section{左偏树}
		没写delete操作,注意初始化内存池和null节点
		\input{leftist-tree.tex}

	\section{Treap}
		包含build, insert和erase,执行时注意初始化内存池和null节点
		\begin{lstlisting}
namespace treap {
	struct node {
		node *left, *right;
		int key;
		int size, count, aux;
		inline node(int _aux) {
			left = right = 0;
			key = size = count = 0;
			aux = _aux;
		}
		inline void update() {
			this->size = this->left->size + this->count + this->right->size;
		}
	};
 
	node *null;
 
	inline void print(node *&x) {
		if (x == null) {
			return;
		}
		print(x->left);
		printf("%d ", x->key);
		print(x->right);
	}
 
	inline node* create(int key) {
		node *x = new node(rand() % INT_MAX);
		x->key = key;
		x->count = x->size = 1;
		x->left = x->right = null;
		return x;
	}
 
	inline void left_rotate(node *&x) {
		node *y = x->right;
		x->right = y->left;
		y->left = x;
		x->update();
		y->update();
		x = y;
	}
 
	inline void right_rotate(node *&x) {
		node *y = x->left;
		x->left = y->right;
		y->right = x;
		x->update();
		y->update();
		x = y;
	}
 
	inline void insert(node *&x, int key) {
		if (x == null) {
			x = create(key);
			return;
		}
		if (x->key == key) {
			x->count++;
		} else if (x->key > key) {
			insert(x->left, key);
			if (x->left->aux < x->aux) {
				right_rotate(x);
			}
		} else {
			insert(x->right, key);
			if (x->right->aux < x->aux) {
				left_rotate(x);
			}
		}
		x->update();
	}
 
	inline void erase(node *&x, int key) {
		if (x == null) {
			return;
		}
		if (x->key == key) {
			if (x->count > 1) {
				x->count--;
			} else if (x->left == null && x->right == null) {
				delete(x);
				x = null;
				return;
			} else if (x->left->aux < x->right->aux) {
				right_rotate(x);
				erase(x->right, key);
			} else {
				left_rotate(x);
				erase(x->left, key);
			}
		} else if (x->key > key) {
			erase(x->left, key);
		} else {
			erase(x->right, key);
		}
		x->update();
	}
 
	inline void prepare() {
		null = new node(INT_MAX);
	}
}
\end{lstlisting}


	\section{线段树}
		包含建树和区间操作样例,没有写具体操作
		\input{segment-tree.tex}

	\section{轻重链剖分}
		包含BFS剖分过程,线段树部分视题目而定
		\input{decomposition.tex}
	
	\section{KMP}
		\input{SCL_KMP.tex}
	
	\section{扩展KMP}
		传入字符串s和长度N,next[i]=LCP(s, s[i..N-1])
		\input{z.tex}

	\section{AC自动机}
		\subsection{Logic\_IU}
			包含建trie和构造自动机的过程
			\input{acautomaton.tex}
		\subsection{shytangyuan}
			\input{aczidongji_shy.tex}

	\section{后缀数组}
		\subsection{Logic\_IU}
			对于串a求SA,长度为N,M为元素值范围,height[i]=LCP(suf[rank[i]],suf[rank[i]-1])
			\input{suffix-array.tex}
		\subsection{shytangyuan}
			\input{后缀数组_shy.tex}

\chapter{杂}
	\section{FFT}
		\input{SCL_FFT_DVCQ.tex}

	\section{中国剩余定理}
	    包括扩展欧几里得,求逆元,和保证除数互质条件下的CRT
	    \input{crt.tex}

	\section{Pollard's Rho$+$Miller-Rabbin}
	    大数分解和素性判断
	    \input{rho.tex}
    
    \section{长方体表面两点最短距离}
        \input{长方体表面两点最短距离.tex}

	\section{字符串的最小表示}
		\subsection{Logic\_IU}
			\input{字符串的最小表示.tex}
		\subsection{tEJtM}
			\input{SCL_str_cyc_min_rep.tex}
    
    \section{牛顿迭代开根号}
        \input{牛顿迭代开根号.tex}

    \section{求某年某月某日星期几}
        \input{求某年某月某日星期几.tex}
	
	\section{A*}
		\input{astar_shy.tex}

	\section{Dancing Links}
		\input{Dancing.tex}

	\section{弦图判定}
		\input{弦图判定.tex}
	
	\section{弦图求团数}
		\input{弦图求团数.tex}

	\section{有根树的同构}
		\input{有根树的同构.tex}

	\section{vimrc}
		\begin{lstlisting}
set cin nu mouse=a nobk hls si go= ts=4 sts=4 sw=4

nmap <C-A> ggVG
vmap <C-C> "+y

syntax on

nmap<F4> : !gedit % <CR>
nmap<F3> : vnew %<.in <CR>
nmap<F5> : !./%< <CR>
nmap<F9> : !g++ % -o %< -Wall <CR>
nmap<F8> : !time ./%< < %<.in <CR>
nmap<F10> : !javac % <CR>
nmap<F6> : !java %< < %<.in <CR>
\end{lstlisting}

\end{document}
