\begin{enumerate}\setlength{\itemsep}{-\itemsep}
	\item 定义:设$m > 1$,,$(a, m) = 1$,使得$a^r \equiv 1(\pmod m)$成立的最小的$r$,称为$a$对模$m$的阶,
	记作$\delta_m(a)$。
	\item 定义:设$m$是正整数,$a$是整数,若$a$模$m$的阶等于$\Phi(m)$,则称$a$为模$m$的一个原根.
	\item 定理:如果模$m$有原根,那么它一共有$\Phi(\Phi(m))$个原根。
	\item 定理:如果$m > 1$,$(a, m) = 1$,$a^n \equiv 1(\pmod m)$,则$\delta_m(a)|n$。
	\item 定理:模$m$有原根的充要条件是$m = 2, 4, p^a, 2p^a$。
	\item 求模素数$p$原根的方法:对$p - 1$素因子分解,即$p - 1 = p_1^{k_1}p_2{k_2}\cdots p_n^{k_n}$,若
	$g^\frac{p - 1}{p_i}$恒成立,那么$g$为$p$的一个原根。(对于合数求原根,只需要将$p - 1$换成$\Phi(p)$即可)
\end{enumerate}
