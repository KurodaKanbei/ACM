\documentclass[landscape]{book}
%\usepackage{ctex}
\usepackage{bm}
%\usepackage[fleqn]{amsmath}
\usepackage{harpoon}
\usepackage{fontspec}
\usepackage{listings}
\usepackage[left=1.5cm, right=1.5cm]{geometry}
\usepackage{setspace}
\usepackage{bm}
\usepackage{cmap}
\usepackage{cite}
\usepackage{float}
\usepackage{xeCJK}
\usepackage{amsthm}
\usepackage{amsmath}
\usepackage{amssymb}
\usepackage{setspace}
\usepackage{enumerate}
\usepackage{indentfirst}
\allowdisplaybreaks
\oddsidemargin -0.1 true cm
\if@twoside
	\evensidemargin -0.1 true cm
\fi
%\setlength{\parindent}{0em}
%\setlength{\mathindent}{0pt}
\newfontfamily\Courier{Courier New}
\lstset{
	language=C++,
	tabsize=4,
	numbers=left,
	breaklines=tr,
	extendedchars=false
	xleftmargin=0em,
	xrightmargin=0em,
	aboveskip=1em,
	numberstyle=\small\Courier,
    basicstyle=\small\Courier
}
\begin{document}
	\title{\textbf{\LARGE{Standard Code Library}}}
	\author{Shanghai Jiao Tong University}
	\date{October, 2015}
	\maketitle
	\tableofcontents
	\begin{spacing}{1.0}
	\chapter{数论算法}
		\section{快速数论变换}
			使用条件及注意事项:$mod$必须要是一个形如$a2^b + 1$的数,$prt$表示$mod$的原根。
			\begin{lstlisting}
const int mod = 998244353;
const int prt = 3;
int prepare(int n) {
	int len = 1;
	for (; len <= 2 * n; len <<= 1);
	for (int i = 0; i <= len; i++) {
		e[0][i] = fpm(prt, (mod - 1) / len * i, mod);
		e[1][i] = fpm(prt, (mod - 1) / len * (len - i), mod);
	}
	return len;
}
void DFT(int *a, int n, int f) {
	for (int i = 0, j = 0; i < n; i++) {
		if (i > j) std::swap(a[i], a[j]);
		for (int t = n >> 1; (j ^= t) < t; t >>= 1);
	}
	for (int i = 2; i <= n; i <<= 1)
		for (int j = 0; j < n; j += i)
			for (int k = 0; k < (i >> 1); k++) {
				int A = a[j + k];
				int B = (long long)a[j + k + (i >> 1)] * e[f][n / i * k] % mod;
				a[j + k] = (A + B) % mod;
				a[j + k + (i >> 1)] = (A - B + mod) % mod;
			}
	if (f == 1) {
		long long rev = fpm(n, mod - 2, mod);
		for (int i = 0; i < n; i++) {
			a[i] = (long long)a[i] * rev % mod;
		}
	}
}
\end{lstlisting}

		\section{多项式求逆}
			使用条件及注意事项:求一个多项式在模意义下的逆元。
			\begin{lstlisting}
void getInv(int *a, int *b, int n) {
	static int tmp[MAXN];
	std::fill(b, b + n, 0);
	b[0] = fpm(a[0], mod - 2, mod);
	for (int c = 1; c <= n; c <<= 1) {
		for (int i = 0; i < c; i++) tmp[i] = a[i];
		std::fill(b + c, b + (c << 1), 0);
		std::fill(tmp + c, tmp + (c << 1), 0);
		DFT(tmp, c << 1, 0);
		DFT(b, c << 1, 0);
		for (int i = 0; i < (c << 1); i++) {
			b[i] = (long long)(2 - (long long)tmp[i] * b[i] % mod + mod) * b[i] % mod;
		}
		DFT(b, c << 1, 1);
		std::fill(b + c, b + (c << 1), 0);
	}
}
\end{lstlisting}

		\section{中国剩余定理}
			使用条件及注意事项:模数可以不互质。
			\begin{lstlisting}
bool solve(int n, std::pair<long long, long long> input[],
                  std::pair<long long, long long> &output) {
    output = std::make_pair(1, 1);
    for (int i = 0; i < n; ++i) {
        long long number, useless;
        euclid(output.second, input[i].second, number, useless);
        long long divisor = std::__gcd(output.second, input[i].second);
        if ((input[i].first - output.first) % divisor) {
            return false;
        }
        number *= (input[i].first - output.first) / divisor;
        fix(number, input[i].second);
        output.first += output.second * number;
        output.second *= input[i].second / divisor;
        fix(output.first, output.second);
    }
    return true;
}
\end{lstlisting}

		\section{Miller Rabin}
			\begin{lstlisting}
const int BASE[12] = {2, 3, 5, 7, 11, 13, 17, 19, 23, 29, 31, 37};
bool check(const long long &prime, const long long &base) {
    long long number = prime - 1;
    for (; ~number & 1; number >>= 1);
    long long result = power_mod(base, number, prime);
    for (; number != prime - 1 && result != 1 && result != prime - 1; number <<= 1) {
        result = multiply_mod(result, result, prime);
    }
    return result == prime - 1 || (number & 1) == 1;
}
bool miller_rabin(const long long &number) {
    if (number < 2) return false;
    if (number < 4) return true;
    if (~number & 1) return false;
    for (int i = 0; i < 12 && BASE[i] < number; ++i)
        if (!check(number, BASE[i])) return false;
    return true;
}
\end{lstlisting}

		\section{Pollard Rho}
			\begin{lstlisting}
long long pollard_rho(const long long &number, const long long &seed) {
    long long x = rand() % (number - 1) + 1, y = x;
    for (int head = 1, tail = 2; ; ) {
        x = multiply_mod(x, x, number);
        x = add_mod(x, seed, number);
        if (x == y) {
            return number;
        }
        long long answer = std::__gcd(abs(x - y), number);
        if (answer > 1 && answer < number) {
            return answer;
        }
        if (++head == tail) {
            y = x;
            tail <<= 1;
        }
    }
}
void factorize(const long long &number, std::vector<long long> &divisor) {
    if (number > 1) {
        if (miller_rabin(number)) {
            divisor.push_back(number);
        } else {
            long long factor = number;
            for (; factor >= number;
                   factor = pollard_rho(number, rand() % (number - 1) + 1));
            factorize(number / factor, divisor);
            factorize(factor, divisor);
        }
    }
}
\end{lstlisting}

		\section{坚固的逆元}
			\begin{lstlisting}
long long inverse(const long long &x, const long long &mod) {
    if (x == 1) {
        return 1;
    } else {
        return (mod - mod / x) * inverse(mod % x, mod) % mod;
    }
}
\end{lstlisting}

		\section{直线下整点个数}
			\begin{lstlisting}
long long solve(const long long &n, const long long &a,
                const long long &b, const long long &m) {
    if (b == 0) return n * (a / m);
    if (a >= m) return n * (a / m) + solve(n, a % m, b, m);
    if (b >= m) return (n - 1) * n / 2 * (b / m) + solve(n, a, b % m, m);
    return solve((a + b * n) / m, (a + b * n) % m, m, b);
}
\end{lstlisting}

	\chapter{数值算法}
		\section{快速傅立叶变换}
			\input{Source/Numerical-Algorithm/Fast-Fourier-Transform.tex}
		\section{单纯形法求解线性规划}
			使用条件及注意事项:返回结果为$max\{c_{1 \times m} \cdot x_{m \times 1} \ | \ x_{m \times 1} \geq 0_{m \times 1}, a_{n \times m} \cdot x_{m \times 1} \leq b_{n \times 1}\}$
			\begin{lstlisting}
std::vector<double> solve(const std::vector<std::vector<double> > &a, 
                          const std::vector<double> &b, const std::vector<double> &c) {
    int n = (int)a.size(), m = (int)a[0].size() + 1;
    std::vector<std::vector<double> > value(n + 2, std::vector<double>(m + 1));
    std::vector<int> index(n + m);
    int r = n, s = m - 1;
    for (int i = 0; i < n + m; ++i) {
        index[i] = i;
    }
    for (int i = 0; i < n; ++i) {
        for (int j = 0; j < m - 1; ++j) {
            value[i][j] = -a[i][j];
        }
        value[i][m - 1] = 1;
        value[i][m] = b[i];
        if (value[r][m] > value[i][m]) {
            r = i;
        }
    }
    for (int j = 0; j < m - 1; ++j) {
        value[n][j] = c[j];
    }
    value[n + 1][m - 1] = -1;
    for (double number; ; ) {
        if (r < n) {
            std::swap(index[s], index[r + m]);
            value[r][s] = 1 / value[r][s];
            for (int j = 0; j <= m; ++j) {
                if (j != s) {
                    value[r][j] *= -value[r][s];
                }
            }
            for (int i = 0; i <= n + 1; ++i) {
                if (i != r) {
                    for (int j = 0; j <= m; ++j) {
                        if (j != s) {
                            value[i][j] += value[r][j] * value[i][s];
                        }
                    }
                    value[i][s] *= value[r][s];
                }
            }
        }
        r = s = -1;
        for (int j = 0; j < m; ++j) {
            if (s < 0 || index[s] > index[j]) {
                if (value[n + 1][j] > eps || value[n + 1][j] > -eps && value[n][j] > eps) {
                    s = j;
                }
            }
        }
        if (s < 0) {
            break;
        }
        for (int i = 0; i < n; ++i) {
            if (value[i][s] < -eps) {
                if (r < 0
                || (number = value[r][m] / value[r][s] - value[i][m] / value[i][s]) < -eps
                || number < eps && index[r + m] > index[i + m]) {
                     r = i;
                }
            }
        }
        if (r < 0) {
            //    Solution is unbounded.
            return std::vector<double>();
        }
    }
    if (value[n + 1][m] < -eps) {
        //    No solution.
        return std::vector<double>();
    }
    std::vector<double> answer(m - 1);
    for (int i = m; i < n + m; ++i) {
        if (index[i] < m - 1) {
            answer[index[i]] = value[i - m][m];
        }
    }
    return answer;
}
\end{lstlisting}

		\section{自适应辛普森}
			\input{Source/Numerical-Algorithm/Adaptive-Simpson.tex}
	\chapter{数据结构}
		\section{Splay普通操作版}
			使用条件及注意事项:\par
			\begin{enumerate}
				\item 插入$x$数
				\item 删除$x$数(若有多个相同的数,因只删除一个)
				\item 查询$x$数的排名(若有多个相同的数,因输出最小的排名)
				\item 查询排名为$x$的数
				\item 求$x$的前驱(前驱定义为小于$x$,且最大的数)
				\item 求$x$的后继(后继定义为大于$x$,且最小的数)
			\end{enumerate}
			\begin{lstlisting}
int pred(int x) {
	splay(x, -1);
	for (x = c[x][0]; c[x][1]; x = c[x][1]);
	return x;
}
int succ(int x) {
	splay(x, -1);
	for (x = c[x][1]; c[x][0]; x = c[x][0]);
	return x;
}
void remove(int x) {
	if (b[x] > 1) {b[x]--; splay(x, -1); return;}
	splay(x, -1);
	if (!c[x][0] && !c[x][1]) rt = 0;
	else if (c[x][0] && !c[x][1]) f[rt = c[x][0]] = -1;
	else if (!c[x][0] && c[x][1]) f[rt = c[x][1]] = -1;
	else{
		int t = pred(x); f[rt = c[x][0]] = -1;
		c[t][1] = c[x][1]; f[c[x][1]] = t;
		splay(c[x][1], -1);
	}
	c[x][0] = c[x][1] = f[x] = d[x] = s[x] = b[x] = 0;
}
int find(int z) {
	int x=rt;
	while (d[x]!=z)
		if (c[x][d[x]<z]) x=c[x][d[x]<z];
		else break;
	return x;
}
void insert(int z) {
	if (!rt) {
		f[rt = ++size] = -1;
		d[size] = z; b[size] = 1;
		splay(size, -1);
		return;
	}
	int x = find(z);
	if (d[x] == z) {
		b[x]++;
		splay(x, -1);
		return;
	}
	c[x][d[x]<z] = ++size; f[size] = x;
	d[size] = z; b[size] = s[size] = 1;
	splay(size, -1);
}
int select(int z) {
	int x = rt;
	while (z < s[c[x][0]] + 1 || z > s[c[x][0]] + b[x])
		if (z > s[c[x][0]] + b[x]) {
			z -= s[c[x][0]] + b[x];
			x = c[x][1];
		}
	  	else x = c[x][0];
	return x;
}
int main() {
	scanf("%d",&n);
	for (int i = 1; i <= n; i++) {
		int opt, x;
		scanf("%d%d", &opt, &x);
		if (opt == 1) insert(x);
		else if (opt == 2) remove(find(x)); //删除x数(若有多个相同的数,因只删除一个)
		else if (opt == 3) { // 查询x数的排名(若有多个相同的数,因输出最小的排名)
			insert(x);
			printf("%d\n", s[c[find(x)][0]] + 1);
			remove(find(x));
		}
		else if (opt == 4) printf("%d\n",d[select(x)]);
		else if (opt == 5) {
			insert(x);
			printf("%d\n", d[pred(find(x))]);
			remove(find(x));
		}
		else if (opt == 6) {
			insert(x);
			printf("%d\n", d[succ(find(x))]);
			remove(find(x));
		}
	}
	return 0;
}
\end{lstlisting}

		\section{Splay区间操作版}
			使用条件及注意事项:\par
			这是为NOI2005维修数列的代码,仅供区间操作用的splay参考。
			\begin{lstlisting}
const int INF = 100000000;
const int Maxspace = 500000;
struct SplayNode{
	int ls, rs, zs, ms;
	SplayNode() {
		ms = 0;
		ls = rs = zs = -INF;
	}
	SplayNode(int d) {
		ms = zs = ls = rs = d;
	}
	SplayNode operator +(const SplayNode &p)const {
		SplayNode ret;
		ret.ls = max(ls, ms + p.ls);
		ret.rs = max(rs + p.ms, p.rs);
		ret.zs = max(rs + p.ls, max(zs, p.zs));
		ret.ms = ms + p.ms;
		return ret;
	}
}t[MAXN], d[MAXN];
int n, m, rt, top, a[MAXN], f[MAXN], c[MAXN][2], g[MAXN], h[MAXN], z[MAXN];
bool r[MAXN], b[MAXN];
void makesame(int x, int s) {
	if (!x) return;
	b[x] = true;
	d[x] = SplayNode(g[x] = s);
	t[x].zs = t[x].ms = g[x] * h[x];
	t[x].ls = t[x].rs = max(g[x], g[x] * h[x]);
}
void makerev(int x) {
	if (!x) return;
	r[x] ^= 1;
	swap(c[x][0], c[x][1]);
	swap(t[x].ls, t[x].rs);
}
void pushdown(int x) {
	if (!x) return;
	if (r[x]) {
		makerev(c[x][0]);
		makerev(c[x][1]);
		r[x]=0;
	}
	if (b[x]) {
		makesame(c[x][0],g[x]);
		makesame(c[x][1],g[x]);
		b[x]=g[x]=0;
	}
}
void updata(int x) {
	if (!x) return;
	h[x]=h[c[x][0]]+h[c[x][1]]+1;
	t[x]=t[c[x][0]]+d[x]+t[c[x][1]];
}
void rotate(int x,int k) {
	pushdown(x);pushdown(c[x][k]);
	int y = c[x][k]; c[x][k] = c[y][k^1]; c[y][k^1] = x;
	if (f[x] != -1) c[f[x]][c[f[x]][1] == x] = y; else rt = y;
	f[y] = f[x]; f[x] = y; f[c[x][k]] = x;
	updata(x); updata(y);
}
void splay(int x, int s) {
	while (f[x] != s) {
		if (f[f[x]]!=s) {
			pushdown(f[f[x]]);
			rotate(f[f[x]], (c[f[f[x]]][1] == f[x]) ^ r[f[f[x]]]);
		}
		pushdown(f[x]);
		rotate(f[x], (c[f[x]][1]==x) ^ r[f[x]]);
	}
}
void build(int &x,int l,int r) {
	if (l > r) {x = 0; return;}
	x = z[top--];
	if (l < r) {
		build(c[x][0],l,(l+r>>1)-1);
		build(c[x][1],(l+r>>1)+1,r);
	}
	f[c[x][0]] = f[c[x][1]] = x;
	d[x] = SplayNode(a[l+r>>1]);
	updata(x);
}
void init() {
	d[0] = SplayNode();
	f[rt=2] = -1;
	f[1] = 2; c[2][0] = 1;
	int x;
	build(x,1,n);
	c[1][1] = x; f[x] = 1;
	splay(x, -1);
}
int find(int z) {
	int x = rt; pushdown(x);
	while (z != h[c[x][0]] + 1) {
		if (z > h[c[x][0]] + 1) {
			z -= h[c[x][0]] + 1;
			x = c[x][1];
		}
		else x = c[x][0];
		pushdown(x);
	}
	return x;
}
void getrange(int &x,int &y) {
	y = x + y - 1;
	x = find(x);
	y = find(y + 2);
	splay(y, -1);
	splay(x, y);
}
void recycle(int x) {
	if (!x) return;
	recycle(c[x][0]);
	recycle(c[x][1]);
	z[++top]=x;
	t[x] = d[x] = SplayNode();
	r[x] = b[x] = g[x] = f[x] = h[x] = 0;
	c[x][0] = c[x][1]=0;
}
int main() {
	scanf("%d%d",&n,&m);
	for (int i = 1; i <= n; i++) scanf("%d",a+i);
	for (int i = Maxspace; i>=3; i--) z[++top] = i;
	init();
	for (int i = 1; i <= m; i++) {
		char op[10];
		int x, y, tmp;
		scanf("%s", op);
		if (!strcmp(op, "INSERT")) {
			scanf("%d%d", &x, &y);
			n += y;
			if (!y) continue;
			for (int i = 1; i <= y; i++) scanf("%d",a+i);
			build(tmp, 1, y);
			x = find(x + 1); pushdown(x);
			if (!c[x][1]) {c[x][1] = tmp; f[tmp] = x;}
			else{
				x = c[x][1]; pushdown(x);
				while (c[x][0]) {
					x = c[x][0];
					pushdown(x);
				}
				c[x][0] = tmp; f[tmp] = x;
			}
			splay(tmp, -1);
		}
		else if (!strcmp(op, "DELETE")) {
			scanf("%d%d", &x, &y); n -= y;
			if (!y) continue;
			getrange(x, y);
			int k = (c[y][0] == x);
			recycle(c[x][k]);
			f[c[x][k]] = 0;
			c[x][k] = 0;
			splay(x, -1);
		}
		else if (!strcmp(op, "REVERSE")) {
			scanf("%d%d", &x, &y);
			if (!y) continue;
			getrange(x, y);
			int k = (c[y][0]==x);
			makerev(c[x][k]);
			splay(c[x][k], -1);
		}
		else if (!strcmp(op, "GET-SUM")) {
			scanf("%d%d", &x, &y);
			if (!y) {
				printf("0\n");
				continue;
			}
			getrange(x,y);
			int k = (c[y][0] == x);
			printf("%d\n", t[c[x][k]].ms);
			splay(c[x][k], -1);
		}
		else if (!strcmp(op, "MAX-SUM")) {
			x = 1; y = n;
			getrange(x, y);
			int k = (c[y][0] == x);
			printf("%d\n", t[c[x][k]].zs);
			splay(c[x][k], -1);
		}
		else if (!strcmp(op, "MAKE-SAME")) {
			scanf("%d%d%d", &x, &y, &tmp);
			if (!y) continue;
			getrange(x, y);
			int k = (c[y][0] == x);
			makesame(c[x][k], tmp);
			splay(c[x][k], -1);
		}
	}
	return 0;
}
\end{lstlisting}

		\section{坚固的Treap}
			使用条件及注意事项:题目来源UVA 12358
			\input{Source/Data-Structure/Persistent-Treap}
		\section{k-d树}
			使用条件及注意事项:这是求$k$远点的代码,要求$k$近点的话把堆的比较函数改一改,把朝左儿子或者是右儿子
			的方向改一改。
			\begin{lstlisting}
struct Heapnode{
	long long d;
	int pos;
	bool operator <(const Heapnode &p)const {
		return d > p.d || (d == p.d && pos < p.pos);
	}
};

struct MsgNode{
	int xmin, xmax, ymin, ymax;
	MsgNode() {}
	MsgNode(const Point &a) : xmin(a.x), xmax(a.x), ymin(a.y), ymax(a.y) {}
	long long dist(const Point &a) {
		int dx = std::max(std::abs(a.x - xmin), std::abs(a.x - xmax));
		int dy = std::max(std::abs(a.y - ymin), std::abs(a.y - ymax));
		return (long long)dx * dx + (long long)dy * dy;
	}
	MsgNode operator +(const MsgNode &rhs)const {
		MsgNode ret;
		ret.xmin = std::min(xmin, rhs.xmin);
		ret.xmax = std::max(xmax, rhs.xmax);
		ret.ymin = std::min(ymin, rhs.ymin);
		ret.ymax = std::max(ymax, rhs.ymax);
		return ret;
	}
};

struct TNode{
	int l, r;
	Point p;
	MsgNode d;
}tree[MAXN];

void buildtree(int &rt, int l, int r, int pivot) {
	if (l > r) return;
	rt = ++size;
	int mid = l + r >> 1;
	if (pivot == 1) std::nth_element(p + l, p + mid, p + r + 1, cmpx);
	if (pivot == 0) std::nth_element(p + l, p + mid, p + r + 1, cmpy);
	tree[rt].d = MsgNode(tree[rt].p = p[mid]);
	buildtree(tree[rt].l, l, mid - 1, pivot ^ 1);
	buildtree(tree[rt].r, mid + 1, r, pivot ^ 1);
	if (tree[rt].l) tree[rt].d = tree[rt].d + tree[tree[rt].l].d;
	if (tree[rt].r) tree[rt].d = tree[rt].d + tree[tree[rt].r].d;
}

void query(int rt, const Point &a, int k, int pivot) {
	Heapnode now = (Heapnode){dist(a, tree[rt].p), tree[rt].p.pos};
	if (heap.size() < k) heap.push(now);
	else if (now < heap.top()) {heap.pop(); heap.push(now);}
	int lson = tree[rt].l, rson = tree[rt].r;
	if (pivot == 1 && cmpx(a, tree[rt].p)) std::swap(lson, rson);
	if (pivot == 0 && cmpy(a, tree[rt].p)) std::swap(lson, rson);
	if (lson && (heap.size() < k || tree[lson].d.dist(a) >= heap.top().d)) query(lson, a, k, pivot ^ 1);
	if (rson && (heap.size() < k || tree[rson].d.dist(a) >= heap.top().d)) query(rson, a, k, pivot ^ 1);
}

int main() {
	for (int i = 1; i <= q; i++) {
		int k;
		Point now;
		now.read();
		scanf("%d", &k);
		while (!heap.empty()) heap.pop();
		query(rt, now, k, 1);
		printf("%d\n", heap.top().pos);
	}
	return 0;
}
\end{lstlisting}

		\section{树链剖分}
			\subsection{点操作版本}
				使用条件及注意事项:树上最大(非空)子段和,注意一条路径询问的时候信息统计的顺序。
				\begin{lstlisting}
struct Node{
	int asum, lsum, rsum, zsum;
	Node() {
		asum = 0;
		lsum = -INF;
		rsum = -INF;
		zsum = -INF;
	}
	Node(int d) : asum(d), lsum(d), rsum(d), zsum(d) {}
	Node operator +(const Node &rhs)const {
		Node ret;
		ret.asum = asum + rhs.asum;
		ret.lsum = std::max(lsum, asum + rhs.lsum);
		ret.rsum = std::max(rsum + rhs.asum, rhs.rsum);
		ret.zsum = std::max(zsum, rhs.zsum);
		ret.zsum = std::max(ret.zsum, rsum + rhs.lsum);
		return ret;
	}
}tree[MAXN * 6];

int n, q, cnt, tot, h[MAXN], d[MAXN], t[MAXN], f[MAXN], s[MAXN], z[MAXN], w[MAXN], o[MAXN], a[MAXN];
std::pair<bool, int> flag[MAXN * 6];

void addedge(int x, int y) {
	cnt++; e[cnt] = (Edge){y, h[x]}; h[x] = cnt;
	cnt++; e[cnt] = (Edge){x, h[y]}; h[y] = cnt;
}

void makesame(int n, int l, int r, int d) {
	flag[n] = std::make_pair(true, d);
	tree[n].asum = d * (r - l + 1);
	if (d > 0) {
		tree[n].lsum = d * (r - l + 1);
		tree[n].rsum = d * (r - l + 1);
		tree[n].zsum = d * (r - l + 1);
	}
	else{
		tree[n].lsum = d;
		tree[n].rsum = d;
		tree[n].zsum = d;
	}
}

void pushdown(int n, int l, int r) {
	if (flag[n].first) {
		makesame(n << 1, l, l + r >> 1, flag[n].second);
		makesame(n << 1 ^ 1, (l + r >> 1) + 1, r, flag[n].second);
		flag[n] = std::make_pair(false, 0);
	}
}

void modify(int n, int l, int r, int x, int y, int d) {
	if (x <= l && r <= y) {
		makesame(n, l, r, d);
		return;
	}
	pushdown(n, l, r);
	if ((l + r >> 1) < x) modify(n << 1 ^ 1, (l + r >> 1) + 1, r, x, y, d);
	else if ((l + r >> 1) + 1 > y) modify(n << 1, l, l + r >> 1, x, y, d);
	else{
		modify(n << 1, l, l + r >> 1, x, y, d);
		modify(n << 1 ^ 1, (l + r >> 1) + 1, r, x, y, d);
	}
	tree[n] = tree[n << 1] + tree[n << 1 ^ 1];
}

Node query(int n, int l, int r, int x, int y) {
	if (x <= l && r <= y) return tree[n];
	pushdown(n, l, r);
	if ((l + r >> 1) < x) return query(n << 1 ^ 1, (l + r >> 1) + 1, r, x, y);
	else if ((l + r >> 1) + 1 > y) return query(n << 1, l, l + r >> 1, x, y);
	else{
		Node left = query(n << 1, l, l + r >> 1, x, y);
		Node right = query(n << 1 ^ 1, (l + r >> 1) + 1, r, x, y);
		return left + right;
	}
}

void modify(int x, int y, int val) {
	int fx = t[x], fy = t[y];
	while (fx != fy) {
		if (d[fx] > d[fy]) {
			modify(1, 1, n, w[fx], w[x], val);
			x = f[fx]; fx = t[x];
		}
		else{
			modify(1, 1, n, w[fy], w[y], val);
			y = f[fy]; fy = t[y];
		}
	}
	if (d[x] < d[y]) modify(1, 1, n, w[x], w[y], val);
	else modify(1, 1, n, w[y], w[x], val);
}

Node query(int x, int y) {
	int fx = t[x], fy = t[y];
	Node left = Node(), right = Node();
	while (fx != fy) {
		if (d[fx] > d[fy]) {
			left = query(1, 1, n, w[fx], w[x]) + left;
			x = f[fx]; fx = t[x];
		}
		else{
			right = query(1, 1, n, w[fy], w[y]) + right;
			y = f[fy]; fy = t[y];
		}
	}
	if (d[x] < d[y]) {
		right = query(1, 1, n, w[x], w[y]) + right;
	}
	else{
		left = query(1, 1, n, w[y], w[x]) + left;
	}
	std::swap(left.lsum, left.rsum);
	return left + right;
}

void predfs(int x) {
	s[x] = 1; z[x] = 0;
	for (int i = h[x]; i; i = e[i].next) {
		if (e[i].node == f[x]) continue;
		f[e[i].node] = x;
		d[e[i].node] = d[x] + 1;
		predfs(e[i].node);
		s[x] += s[e[i].node];
		if (s[z[x]] < s[e[i].node]) z[x] = e[i].node;
	}
}

void getanc(int x, int anc) {
	t[x] = anc; w[x] = ++tot; o[tot] = x;
	if (z[x]) getanc(z[x], anc);
	for (int i = h[x]; i; i = e[i].next) {
		if (e[i].node == f[x] || e[i].node == z[x]) continue;
		getanc(e[i].node, e[i].node);
	}
}

void buildtree(int n, int l, int r) {
	if (l == r) {
		tree[n] = Node(a[o[l]]);
		return;
	}
	buildtree(n << 1, l, l + r >> 1);
	buildtree(n << 1 ^ 1, (l + r >> 1) + 1, r);
	tree[n] = tree[n << 1] + tree[n << 1 ^ 1];
}

int main() {
	scanf("%d", &n);
	for (int i = 1; i <= n; i++) scanf("%d", a + i);
	for (int i = 1; i < n; i++) {
		int x, y; scanf("%d%d", &x, &y);
		addedge(x, y);
	}
	predfs(1);
	getanc(1, 1);
	buildtree(1, 1, n);
	scanf("%d", &q);
	for (int i = 1; i <= q; i++) {
		int op, x, y, c;
		scanf("%d", &op);
		if (op == 1) {
			scanf("%d%d", &x, &y);
			Node ret = query(x, y);
			printf("%d\n", std::max(0, ret.zsum));
		}
		else{
			scanf("%d%d%d", &x, &y, &c);
			modify(x, y, c);
		}
	}
	return 0;
}
\end{lstlisting}

			\subsection{链操作版本}
				\begin{lstlisting}
void modify(int x, int y) {
	int fx = t[x], fy = t[y];
	while (fx != fy) {
		if (d[fx] > d[fy]) {
			modify(1, 1, n, w[fx], w[x]);
			x = f[fx]; fx = t[x];
		}
		else{
			modify(1, 1, n, w[fy], w[y]);
			y = f[fy]; fy = t[y];
		}
	}
	if (x != y) {
		if (d[x] < d[y]) modify(1, 1, n, w[z[x]], w[y]);
		else modify(1, 1, n, w[z[y]], w[x]);
	}
}
\end{lstlisting}

		\section{Link-Cut-Tree}
			\begin{lstlisting}
struct MsgNode{
	int leftColor, rightColor, answer;
	MsgNode() {
		leftColor = -1;
		rightColor = -1;
		answer = 0;
	}
	MsgNode(int c) {
		leftColor = rightColor = c;
		answer = 1;
	}
	MsgNode operator +(const MsgNode &p)const {
		if (answer == 0) return p;
		if (p.answer == 0) return *this;
		MsgNode ret;
		ret.leftColor = leftColor;
		ret.rightColor = p.rightColor;
		ret.answer = answer + p.answer - (rightColor == p.leftColor);
		return ret;
	}
}d[MAXN], g[MAXN];
int n, m, c[MAXN][2], f[MAXN], p[MAXN], s[MAXN], flag[MAXN];
bool r[MAXN];
void init(int x, int value) {
	d[x] = g[x] = MsgNode(value);
	c[x][0] = c[x][1] = 0;
	f[x] = p[x] = flag[x] = -1;
	s[x] = 1;
}
void update(int x) {
	s[x] = s[c[x][0]] + s[c[x][1]] + 1;
	g[x] = MsgNode();
	if (c[x][0 ^ r[x]]) g[x] = g[x] + g[c[x][0 ^ r[x]]];
	g[x] = g[x] + d[x];
	if (c[x][1 ^ r[x]]) g[x] = g[x] + g[c[x][1 ^ r[x]]];
}
void makesame(int x, int c) {
	flag[x] = c;
	d[x] = MsgNode(c);
	g[x] = MsgNode(c);
}
void pushdown(int x) {
	if (r[x]) {
		std::swap(c[x][0], c[x][1]);
		r[c[x][0]] ^= 1;
		r[c[x][1]] ^= 1;
		std::swap(g[c[x][0]].leftColor, g[c[x][0]].rightColor);
		std::swap(g[c[x][1]].leftColor, g[c[x][1]].rightColor);
		r[x] = false;
	}
	if (flag[x] != -1) {
		if (c[x][0]) makesame(c[x][0], flag[x]);
		if (c[x][1]) makesame(c[x][1], flag[x]);
		flag[x] = -1;
	}
}
void rotate(int x, int k) {
	pushdown(x); pushdown(c[x][k]);
	int y = c[x][k]; c[x][k] = c[y][k ^ 1]; c[y][k ^ 1] = x;
	if (f[x] != -1) c[f[x]][c[f[x]][1] == x] = y;
	f[y] = f[x]; f[x] = y; f[c[x][k]] = x; std::swap(p[x], p[y]);
	update(x); update(y);
}
void splay(int x, int s = -1) {
	pushdown(x);
	while (f[x] != s) {
		if (f[f[x]] != s) rotate(f[f[x]], (c[f[f[x]]][1] == f[x]) ^ r[f[f[x]]]);
		rotate(f[x], (c[f[x]][1] == x) ^ r[f[x]]);
	}
	update(x);
}
void access(int x) {
	int y = 0;
	while (x != -1) {
		splay(x); pushdown(x);
		f[c[x][1]] = -1; p[c[x][1]] = x;
		c[x][1] = y; f[y] = x; p[y] = -1;
		update(x); x = p[y = x];
	}
}
void setroot(int x) {
	access(x);
	splay(x);
	r[x] ^= 1;
	std::swap(g[x].leftColor, g[x].rightColor);
}
void link(int x, int y) {
	setroot(x);
	p[x] = y;
}
\end{lstlisting}

	\chapter{图论}
		\section{强连通分量}
			\input{Source/Graph-Theory/Strongly-Connected-Components.tex}
		\section{点双连通分量}
			\subsection{坚固的点双连通分量}
\begin{lstlisting}
int n, m, x, y, ans1, ans2, tot1, tot2, flag, size, ind2, dfn[N], low[N], block[M], vis[N];
vector<int> a[N];
pair<int, int> stack[M];
void tarjan(int x, int p) {
	dfn[x] = low[x] = ++ind2;
	for (int i = 0; i < a[x].size(); ++i)
		if (dfn[x] > dfn[a[x][i]] && a[x][i] != p){
			stack[++size] = make_pair(x, a[x][i]);
			if (i == a[x].size() - 1 || a[x][i] != a[x][i + 1])
				if (!dfn[a[x][i]]){
					tarjan(a[x][i], x);
					low[x] = min(low[x], low[a[x][i]]);
					if (low[a[x][i]] >= dfn[x]){
						tot1 = tot2 = 0;
						++flag;
						for (; ; ){
							if (block[stack[size].first] != flag) {
								++tot1;
								block[stack[size].first] = flag;
							}
							if (block[stack[size].second] != flag) {
								++tot1;
								block[stack[size].second] = flag;
							}
							if (stack[size].first == x && stack[size].second == a[x][i])
								break;
							++tot2;
							--size;
						}
						for (; stack[size].first == x && stack[size].second == a[x][i]; --size)
							++tot2;
						if (tot2 < tot1)
							ans1 += tot2;
						if (tot2 > tot1)
							ans2 += tot2;
					}
				}
				else
					low[x] = min(low[x], dfn[a[x][i]]);
		}
}
int main(){
	for (; ; ){
		scanf("%d%d", &n, &m);
		if (n == 0 && m == 0) return 0;
		for (int i = 1; i <= n; ++i) {
			a[i].clear();
			dfn[i] = 0;
		}
		for (int i = 1; i <= m; ++i){
			scanf("%d%d",&x, &y);
			++x, ++y;
			a[x].push_back(y);
			a[y].push_back(x);
		}
		for (int i = 1; i <= n; ++i)
			sort(a[i].begin(), a[i].end());
		ans1 = ans2 = ind2 = 0;
		for (int i = 1; i <= n; ++i)
			if (!dfn[i]) {
				size = 0;
				tarjan(i, 0);
			}
		printf("%d %d\n", ans1, ans2);
	}
	return 0;
}
\end{lstlisting}
\subsection{朴素的点双连通分量}
\begin{lstlisting}
void tarjan(int x){
	dfn[x] = low[x] = ++ind2;
	v[x] = 1;
	for (int i = nt[x]; pt[i]; i = nt[i])
		if (!dfn[pt[i]]){
			tarjan(pt[i]);
			low[x] = min(low[x], low[pt[i]]);
			if (dfn[x] <= low[pt[i]])
				++v[x];
		}
		else
			low[x] = min(low[x], dfn[pt[i]]);
}
int main(){
	for (; ; ){
		scanf("%d%d", &n, &m);
		if (n == 0 && m == 0)
			return 0;
		for (int i = 1; i <= ind; ++i)
			nt[i] = pt[i] = 0;
		ind = n;
		for (int i = 1; i <= ind; ++i)
			last[i] = i;
		for (int i = 1; i <= m; ++i){
			scanf("%d%d", &x, &y);
			++x, ++y;
			edge(x, y), edge(y, x);
		}
		memset(dfn, 0, sizeof(dfn));
		memset(v, 0, sizeof(v));
		ans = num = ind2 = 0;
		for (int i = 1; i <= n; ++i)
			if (!dfn[i]){
				root = i;
				size = 0;
				++num;
				tarjan(i);
				--v[root];
			}
		for (int i = 1; i <= n; ++i)
			if (v[i] + num - 1 > ans)
				ans = v[i] + num - 1;
		printf("%d\n",ans);
	}
	return 0;
}
\end{lstlisting}
	
		\section{2-SAT问题}
			\input{Source/Graph-Theory/Two-Satisfiability.tex}
		\section{二分图最大匹配}
			\subsection{Hungary算法}
				时间复杂度:$\mathcal{O}(V \cdot E)$
				\input{Source/Graph-Theory/Maximum-Matching-Hungary.tex}
			\subsection{Hopcroft Karp算法}
				时间复杂度:$\mathcal{O}(\sqrt{V} \cdot E)$
				\input{Source/Graph-Theory/Maximum-Matching-Hopcroft-Karp.tex}
		\section{二分图最大权匹配}
			时间复杂度:$\mathcal{O}(V^4)$
			\input{Source/Graph-Theory/Maximum-Weight-Matching.tex}
		\section{最大流}
			\subsection{Dinic}
				使用方法以及注意事项:$n$个点,$m$条边,$inf$为一个很大的值,源点$s$,汇点$t$,图中最大点的编号为$t$。\par
				\indent 邻接表:$p$数组记录节点,$nxt$数组指向下一个位置,$c$数组记录可增广量,$h$数组记录表头(初始全为-1)。\par
				\indent 时间复杂度:$\mathcal{O}(V^2 \cdot E)$
				\begin{lstlisting}
int bfs(){
	for (int i = 1;i <= t;i ++) d[i] = -1;
	int l,r;
	q[l = r = 0] = s, d[s] = 0;
	for (;l <= r;l ++)
		for (int k = h[q[l]]; k > -1; k = nxt[k])
			if (d[p[k]] == -1 && c[k] > 0) d[p[k]] = d[q[l]] + 1, q[++ r] = p[k];
	return d[t] > -1 ? 1 : 0;
}
int dfs(int u,int ext){
	if (u == t) return ext;
	int k = w[u],ret = 0;
	for (; k > -1; k = nxt[k], w[u] = k){      //w数组为当前弧 
		if (ext == 0) break;
		if (d[p[k]] == d[u] + 1 && c[k] > 0){
			int flow = dfs(p[k], min(c[k], ext));
			if (flow > 0){
				c[k] -= flow, c[k ^ 1] += flow;
				ret += flow, ext -= flow;     //ret累计增广量,ext记录还可增广的量 
			}
		}
	}
	if (k == -1) d[u] = -1;
	return ret;
}
void dinic() {
	while (bfs()) {
		for (int i = 1; i <= t;i ++) w[i] = h[i];
		dfs(s, inf);
	}
} 
\end{lstlisting}

			\subsection{ISAP}
				\indent 时间复杂度:$\mathcal{O}(V^2 \cdot E)$
				\begin{lstlisting}
int Maxflow_Isap(int s,int t,int n) {
	std::fill(pre + 1, pre + n + 1, 0);
	std::fill(d + 1, d + n + 1, 0);
	std::fill(gap + 1, gap + n + 1, 0);
	for (int i = 1; i <= n; i++) cur[i] = h[i];
	gap[0] = n;
	int u = pre[s] = s, v, maxflow = 0;
	while (d[s] < n) {
		v = n + 1;
		for (int i = cur[u]; i; i = e[i].next)
		if (e[i].flow && d[u] == d[e[i].node] + 1) {
			v = e[i].node; cur[u]=i; break;
		}
		if (v <= n) {
			pre[v] = u; u = v;
			if (v == t) {
				int dflow = INF, p = t; u = s;
				while (p != s) {
					p = pre[p];
					dflow = std::min(dflow, e[cur[p]].flow);
				}
				maxflow += dflow; p = t;
				while (p != s) {
					p = pre[p];
					e[cur[p]].flow -= dflow;
					e[e[cur[p]].opp].flow += dflow;
				}
			}
		}
		else{
			int mindist = n + 1;
			for (int i = h[u]; i; i = e[i].next)
				if (e[i].flow && mindist > d[e[i].node]) {
					mindist = d[e[i].node]; cur[u] = i;
				}
			if (!--gap[d[u]]) return maxflow;
			gap[d[u] = mindist + 1]++; u = pre[u];
		}
	}
	return maxflow;
}
\end{lstlisting}

			\subsection{SAP}
				\indent 时间复杂度:$\mathcal{O}(V^2 \cdot E)$
				\begin{lstlisting}
const int N = 110, M = 30110, INF = 1000000000;//边表不要开小
int n, m, ind, S, T, flow, tot, pt[M], nt[M], last[N], size[M], num[N], h[N], now[N];
void edge(int x, int y, int z){
	last[x] = nt[last[x]] = ++ind;
	pt[ind] = y, size[ind] = z;
}
int aug(int x, int y){
	if (x == T)
		return y;
	int f = y;
	for (int i = now[x]; pt[i]; i = nt[i])
		if (size[i] && h[pt[i]] + 1 == h[x]){
			int z = aug(pt[i], min(f, size[i]));
			f -= z;
			size[i] -= z;
			size[i ^ 1] += z;
			now[x] = i;
			if (h[S] > tot || f == 0)
				return y - f;
		}
	now[x] = nt[x];
	if (--num[h[x]] == 0)
		h[S] = tot + 1;
	++num[++h[x]];
	return y - f;
}
int main(){
	int np, nc;
	for (; scanf("%d%d%d%d", &n, &np, &nc, &m) == 4; ) {
		for (int i = 0; i <= ind; ++i)
			pt[i] = nt[i] = last[i] = size[i] = 0;
		ind = n + 2;
		if (ind % 2 == 0)
			++ind;
		S = n + 1, tot = T = n + 2;
		for (int i = 0; i <= tot; ++i)
			num[i] = h[i] = now[i] = 0;
		for (int i = 1; i <= tot; ++i)
			last[i] = i;
		for (int i = 1; i <= m; ++i){
			int x, y, z;
			for (; getchar() != '('; );
			scanf("%d%*c%d%*c%d", &x, &y, &z);
			++x, ++y;
			edge(x, y, z);
			edge(y, x, 0);
		}
		for (int i = 1; i <= np; ++i) {
			int y, z;
			for (; getchar() != '('; );
			scanf("%d%*c%d", &y, &z);
			++y;
			edge(S, y, z);
			edge(y, S, 0);
		}
		for (int i = 1; i <= nc; ++i) {
			int x, z;
			for (; getchar() != '('; );
			scanf("%d%*c%d", &x, &z);
			++x;
			edge(x, T, z);
			edge(T, x, 0);
		}
		num[0] = tot;
		for (int i = 1; i <= tot; ++i)
			now[i] = nt[i];
		flow = 0;
		for (; h[S] <= T; )
			flow += aug(S, INF);
		printf("%d\n", flow);
	}
	return 0;
}
\end{lstlisting}

		\section{上下界网络流}
			$B(u,v)$表示边$(u,v)$流量的下界,$C(u,v)$表示边$(u,v)$流量的上界,$F(u,v)$表示边$(u,v)$的流量。
			设$G(u,v) = F(u,v) - B(u,v)$,显然有
			$$0 \leq G(u,v) \leq C(u,v)-B(u,v)$$
		\subsection{无源汇的上下界可行流}
			建立超级源点$S^*$和超级汇点$T^*$,对于原图每条边$(u,v)$在新网络中连如下三条边:$S^* \rightarrow v$,容量为$B(u,v)$;$u \rightarrow T^*$,容量为$B(u,v)$;$u \rightarrow v$,容量为$C(u,v) - B(u,v)$。最后求新网络的最大流,判断从超级源点$S^*$出发的边是否都满流即可,边$(u,v)$的最终解中的实际流量为$G(u,v)+B(u,v)$。
		\subsection{有源汇的上下界可行流}
			从汇点$T$到源点$S$连一条上界为$\infty$,下界为$0$的边。按照\textbf{无源汇的上下界可行流}一样做即可,流量即为$T \rightarrow S$边上的流量。
		\subsection{有源汇的上下界最大流}
			\begin{enumerate}
				\item 在\textbf{有源汇的上下界可行流}中,从汇点$T$到源点$S$的边改为连一条上界为$\infty$,下届为$x$的边。$x$满足二分性质,找到最大的$x$使得新网络存在\textbf{无源汇的上下界可行流}即为原图的最大流。
				\item 从汇点$T$到源点$S$连一条上界为$\infty$,下界为$0$的边,变成无源汇的网络。按照\textbf{无源汇的上下界可行流}的方法,建立超级源点$S^*$和超级汇点$T^*$,求一遍$S^* \rightarrow T^*$的最大流,再将从汇点$T$到源点$S$的这条边拆掉,求一次$S \rightarrow T$的最大流即可。
			\end{enumerate}
		\subsection{有源汇的上下界最小流}
			\begin{enumerate}
				\item 在\textbf{有源汇的上下界可行流}中,从汇点$T$到源点$S$的边改为连一条上界为$x$,下界为$0$的边。$x$满足二分性质,找到最小的$x$使得新网络存在\textbf{无源汇的上下界可行流}即为原图的最小流。
				\item 按照\textbf{无源汇的上下界可行流}的方法,建立超级源点$S^*$与超级汇点$T^*$,求一遍$S^* \rightarrow T^*$的最大流,但是注意这一次不加上汇点$T$到源点$S$的这条边,即不使之改为无源汇的网络去求解。求完后,再加上那条汇点$T$到源点$S$上界$\infty$的边。因为这条边下界为$0$,所以$S^*$,$T^*$无影响,再直接求一次$S^* \rightarrow T^*$的最大流。若超级源点$S^*$出发的边全部满流,则$T \rightarrow S$边上的流量即为原图的最小流,否则无解。
			\end{enumerate}
		\section{最小费用最大流}
		\subsection{稀疏图}
			时间复杂度:$\mathcal{O}(V \cdot E^2)$
			\begin{lstlisting}
struct EdgeList {
    int size;
    int last[N];
    int succ[M], other[M], flow[M], cost[M];
    void clear(int n) {
        size = 0;
        std::fill(last, last + n, -1);
    }
    void add(int x, int y, int c, int w) {
        succ[size] = last[x];
        last[x] = size;
        other[size] = y;
        flow[size] = c;
        cost[size++] = w;
    }
} e;
int n, source, target, prev[N];
void add(int x, int y, int c, int w) {
    e.add(x, y, c, w);
    e.add(y, x, 0, -w);
}
bool augment() {
    static int dist[N], occur[N];
    std::vector<int> queue;
    std::fill(dist, dist + n, INT_MAX);
    std::fill(occur, occur + n, 0);
    dist[source] = 0;
    occur[source] = true;
    queue.push_back(source);
    for (int head = 0; head < (int)queue.size(); ++head) {
        int x = queue[head];
        for (int i = e.last[x]; ~i; i = e.succ[i]) {
            int y = e.other[i];
            if (e.flow[i] && dist[y] > dist[x] + e.cost[i]) {
                dist[y] = dist[x] + e.cost[i];
                prev[y] = i;
                if (!occur[y]) {
                    occur[y] = true;
                    queue.push_back(y);
                }
            }
        }
        occur[x] = false;
    }
    return dist[target] < INT_MAX;
}
std::pair<int, int> solve() {
    std::pair<int, int> answer = std::make_pair(0, 0);
    while (augment()) {
        int number = INT_MAX;
        for (int i = target; i != source; i = e.other[prev[i] ^ 1]) {
            number = std::min(number, e.flow[prev[i]]);
        }
        answer.first += number;
        for (int i = target; i != source; i = e.other[prev[i] ^ 1]) {
            e.flow[prev[i]] -= number;
            e.flow[prev[i] ^ 1] += number;
            answer.second += number * e.cost[prev[i]];
        }
    }
    return answer;
}
\end{lstlisting}

		\subsection{稠密图}
			使用条件:费用非负\\
			\indent 时间复杂度:$\mathcal{O}(V \cdot E^2)$
			\begin{lstlisting}
int aug(int no,int res) {
    if(no == t) return cost += pi1 * res,res;
    v[no] = true;
    int flow = 0;
    for(int i = h[no]; ~ i ;i = nxt[i])
		if(cap[i] && !expense[i] && !v[p[i]]) {
			int d = aug(p[i],min(res,cap[i]));
			cap[i] -= d,cap[i ^ 1] += d,flow += d,res -= d;
			if( !res ) return flow;
		}
    return flow;
}
bool modlabel() {
    int d = maxint;
    for(int i = 1;i <= t;++ i)
		if(v[i]) {
			for(int j = h[i]; ~ j ;j = nxt[j])
				if(cap[j] && !v[p[j]] && expense[j] < d) d = expense[j];
		}
    if(d == maxint)return false;
    for(int i = 1;i <= t;++ i)
		if(v[i]) {
			for(int j = h[i];~ j;j = nxt[j])
				expense[j] -= d,expense[j ^ 1] += d;
		}
    pi1 += d;
    return true;
}
void minimum_cost_flow_zkw() {
	cost = 0;
	do{
		do{
			memset(v,false,sizeof v);
		}while (aug(s,maxint));
	}while (modlabel());
}
\end{lstlisting}

		\section{一般图最大匹配}
			时间复杂度:$\mathcal{O}(V^3)$
			\input{Source/Graph-Theory/Maximum-Matching-Blossom.tex}
		\section{无向图全局最小割}
			时间复杂度:$\mathcal{O}(V^3)$\\
			\indent 注意事项:处理重边时,应该对边权累加
			\input{Source/Graph-Theory/Minimum-Cut-Stoer-Wagner.tex}
		\section{最小树形图}
			\input{Source/Graph-Theory/Chu-Liu-Algorithm.tex}
		\section{有根树的同构}
			时间复杂度:$\mathcal{O}(V log V)$
			\input{Source/Graph-Theory/Rooted-Tree-Isomorphism.tex}
		\section{度限制生成树}
			\input{Source/Graph-Theory/Minimum-Spanning-Tree-With-Degree-Limit.tex}
		\section{弦图相关}
			\subsection{弦图的判定}
				\input{Source/Graph-Theory/Chord-Graph-Judgement.tex}
			\subsection{弦图的团数}
				\input{Source/Graph-Theory/Chord-Graph-Group-Counter.tex}
		\section{哈密尔顿回路(ORE性质的图)}
			ORE性质:$$\forall x,y \in V \wedge (x,y) \notin E \ \ s.t. \ \ deg_x+deg_y \geq n$$
			\indent 返回结果:从顶点$1$出发的一个哈密尔顿回路\\
			\indent 使用条件:$n \geq 3$
			\begin{lstlisting}
int left[N], right[N], next[N], last[N];
void cover(int x) {
    left[right[x]] = left[x];
    right[left[x]] = right[x];
}
int adjacent(int x) {
    for (int i = right[0]; i <= n; i = right[i]) {
        if (graph[x][i]) {
            return i;
        }
    }
    return 0;
}
std::vector<int> solve() {
    for (int i = 1; i <= n; ++i) {
        left[i] = i - 1;
        right[i] = i + 1;
    }
    int head, tail;
    for (int i = 2; i <= n; ++i) {
        if (graph[1][i]) {
            head = 1;
            tail = i;
            cover(head);
            cover(tail);
            next[head] = tail;
            break;
        }
    }
    while (true) {
        int x;
        while (x = adjacent(head)) {
            next[x] = head;
            head = x;
            cover(head);
        }
        while (x = adjacent(tail)) {
            next[tail] = x;
            tail = x;
            cover(tail);
        }
        if (!graph[head][tail]) {
            for (int i = head, j; i != tail; i = next[i]) {
                if (graph[head][next[i]] && graph[tail][i]) {
                    for (j = head; j != i; j = next[j]) {
                        last[next[j]] = j;
                    }
                    j = next[head];
                    next[head] = next[i];
                    next[tail] = i;
                    tail = j;
                    for (j = i; j != head; j = last[j]) {
                        next[j] = last[j];
                    }
                    break;
                }
            }
        }
        next[tail] = head;
        if (right[0] > n) {
            break;
        }
        for (int i = head; i != tail; i = next[i]) {
            if (adjacent(i)) {
                head = next[i];
                tail = i;
                next[tail] = 0;
                break;
            }
        }
    }
    std::vector<int> answer;
    for (int i = head; ; i = next[i]) {
        if (i == 1) {
            answer.push_back(i);
            for (int j = next[i]; j != i; j = next[j]) {
                answer.push_back(j);
            }
            answer.push_back(i);
            break;
        }
        if (i == tail) {
            break;
        }
    }
    return answer;
}
\end{lstlisting}

	\chapter{字符串}
		\section{模式串匹配}
			\input{Source/String-Algorithm/Knuth-Morris-Pratt.tex}
		\section{坚固的模式串匹配}
			\begin{lstlisting}
lenA = strlen(A); lenB = strlen(B);
nxt[0] = lenB,nxt[1] = lenB - 1;
for (int i = 0;i <= lenB;i ++)
	if (B[i] != B[i + 1]) {nxt[1] = i; break;}
int j, k = 1, p, L;
for (int i = 2;i < lenB;i ++) {
	p = k + nxt[k] - 1; L = nxt[i - k];
	if (i + L <= p) nxt[i] = L;
	else {
		j = p - i + 1;
		if (j < 0) j = 0;
		while (i + j < lenB && B[i + j] == B[j]) j++;
		nxt[i] = j; k = i;
	}
}
int minlen = lenA <= lenB ? lenA : lenB; ex[0] = minlen;
for (int i = 0;i < minlen;i ++)
	if (A[i] != B[i]) {ex[0] = i; break;}
k = 0;
for (int i = 1;i < lenA;i ++){
	p = k + ex[k] - 1; L = next[i - k];
	if (i + L <= p) ex[i] = L;
	else {
		j = p - i + 1;
		if (j < 0) j = 0;
		while (i + j < lenA && j < lenB && A[i + j] == B[j]) j++;
		ex[i] = j; k = i;
	}
}
\end{lstlisting}

		\section{AC自动机}
			\begin{lstlisting}
int size, c[MAXT][26], f[MAXT], fail[MAXT], d[MAXT];

int alloc() {
	size++;
	std::fill(c[size], c[size] + 26, 0);
	f[size] = fail[size] = d[size] = 0;
	return size;
}

void insert(char *s) {
	int len = strlen(s + 1), p = 1;
	for (int i = 1; i <= len; i++) {
		if (c[p][s[i] - 'a']) p = c[p][s[i] - 'a'];
		else{
			int newnode = alloc();
			c[p][s[i] - 'a'] = newnode;
			d[newnode] = s[i] - 'a';
			f[newnode] = p;
			p = newnode;
		}
	}
}

void buildfail() {
	static int q[MAXT];
	int left = 0, right = 0;
	fail[1] = 0;
	for (int i = 0; i < 26; i++) {
		c[0][i] = 1;
		if (c[1][i]) q[++right] = c[1][i];
	}
	while (left < right) {
		left++;
		int p = fail[f[q[left]]];
		while (!c[p][d[q[left]]]) p = fail[p];
		fail[q[left]] = c[p][d[q[left]]];
		for (int i = 0; i < 26; i++) {
			if (c[q[left]][i]) {
				q[++right] = c[q[left]][i];
			}
		}
	}
	for (int i = 1; i <= size; i++)
		for (int j = 0; j < 26; j++) {
			int p = i;
			while (!c[p][j]) p = fail[p];
			c[i][j] = c[p][j];
		}
}
\end{lstlisting}

		\section{后缀数组}
			\begin{lstlisting}
namespace suffix_array{
	int wa[MAXN], wb[MAXN], ws[MAXN], wv[MAXN];
	bool cmp(int *r, int a, int b, int l) {
		return r[a] == r[b] && r[a + l] == r[b + l];
	}
	void DA(int *r, int *sa, int n, int m) {
		int *x = wa, *y = wb, *t;
		for (int i = 0; i < m; i++) ws[i] = 0;
		for (int i = 0; i < n; i++) ws[x[i] = r[i]]++;
		for (int i = 1; i < m; i++) ws[i] += ws[i - 1];
		for (int i = n - 1; i >= 0; i--) sa[--ws[x[i]]] = i;
		for (int i, j = 1, p = 1; p < n; j <<= 1, m = p) {
			for (p = 0, i = n - j; i < n; i++) y[p++] = i;
			for (i = 0; i < n; i++) if (sa[i] >= j) y[p++] = sa[i] - j;
			for (i = 0; i < n; i++) wv[i] = x[y[i]];
			for (i = 0; i < m; i++) ws[i] = 0;
			for (i = 0; i < n; i++) ws[wv[i]]++;
			for (i = 1; i < m; i++) ws[i] += ws[i-1];
			for (i = n - 1; i >= 0; i--) sa[--ws[wv[i]]] = y[i];
			for (t = x, x = y, y = t, p = 1, x[sa[0]] = 0, i = 1; i < n; i++)
				x[sa[i]] = cmp(y, sa[i - 1], sa[i], j) ? p - 1 : p++;
		}
	}
	void getheight(int *r, int *sa, int *rk, int *h, int n) {
		for (int i = 1; i <= n; i++) rk[sa[i]] = i;
		for (int i = 0, j, k = 0; i < n; h[rk[i++]] = k)
			for (k ? k-- : 0, j = sa[rk[i] - 1]; r[i + k] == r[j + k]; k++);
	}
};
\end{lstlisting}

		\section{广义后缀自动机}
			\begin{lstlisting}
// Generalized Suffix Automaton
void add(int x, int &last) {
	int lastnode = last;
	if (c[lastnode][x]) {
		int nownode = c[lastnode][x];
		if (l[nownode] == l[lastnode] + 1) last = nownode;
		else{
			int auxnode = ++size; l[auxnode] = l[lastnode] + 1;
			for (int i = 0; i < 26; i++) c[auxnode][i] = c[nownode][i];
			f[auxnode] = f[nownode]; f[nownode] = auxnode;
			for (; lastnode && c[lastnode][x] == nownode; lastnode = f[lastnode]) {
				c[lastnode][x] = auxnode;
			}
			last = auxnode;
		}
	}
	else{
		int newnode = ++size; l[newnode] = l[lastnode] + 1;
		for (; lastnode && !c[lastnode][x]; lastnode = f[lastnode]) c[lastnode][x] = newnode;
		if (!lastnode) f[newnode] = 1;
		else{
			int nownode = c[lastnode][x];
			if (l[lastnode] + 1 == l[nownode]) f[newnode] = nownode;
			else{
				int auxnode = ++size; l[auxnode] = l[lastnode] + 1;
				for (int i = 0; i < 26; i++) c[auxnode][i] = c[nownode][i];
				f[auxnode] = f[nownode]; f[nownode] = f[newnode] = auxnode;
				for (; lastnode && c[lastnode][x] == nownode; lastnode = f[lastnode]) {
					c[lastnode][x] = auxnode;
				}
			}
		}
		last = newnode;
	}
}
\end{lstlisting}

		\section{Manacher算法}
			\begin{lstlisting}
void manacher(char *text, int length) {
    palindrome[0] = 1;
    for (int i = 1, j = 0; i < length; ++i) {
        if (j + palindrome[j] <= i) {
            palindrome[i] = 0;
        } else {
            palindrome[i] = std::min(palindrome[(j << 1) - i], j + palindrome[j] - i);
        }
        while (i - palindrome[i] >= 0 && i + palindrome[i] < length 
                && text[i - palindrome[i]] == text[i + palindrome[i]]) {
            palindrome[i]++;
        }
        if (i + palindrome[i] > j + palindrome[j]) {
            j = i;
        }
    }
}
\end{lstlisting}

		\section{回文树}
			\begin{lstlisting}
struct Palindromic_Tree{
	int nTree, nStr, last, c[MAXT][26], fail[MAXT], r[MAXN], l[MAXN], s[MAXN];
	int allocate(int len) {
		l[nTree] = len;
		r[nTree] = 0;
		fail[nTree] = 0;
		memset(c[nTree], 0, sizeof(c[nTree]));
		return nTree++;
	}
	void init() {
		nTree = nStr = 0;
		int newEven = allocate(0);
		int newOdd = allocate(-1);
		last = newEven;
		fail[newEven] = newOdd;
		fail[newOdd] = newEven;
		s[0] = -1;
	}
	void add(int x) {
		s[++nStr] = x;
		int nownode = last;
		while (s[nStr - l[nownode] - 1] != s[nStr]) nownode = fail[nownode];
		if (!c[nownode][x]) {
			int newnode = allocate(l[nownode] + 2), &newfail = fail[newnode];
			newfail = fail[nownode];
			while (s[nStr - l[newfail] - 1] != s[nStr]) newfail = fail[newfail];
			newfail = c[newfail][x];
			c[nownode][x] = newnode;
		}
		last = c[nownode][x];
		r[last]++;
	}
	void count() {
		for (int i = nTree - 1; i >= 0; i--) {
			r[fail[i]] += r[i];
		}
	}
}
\end{lstlisting}

		\section{循环串最小表示}
			\input{Source/String-Algorithm/Minimum-Circular-Representation.tex}
	\chapter{计算几何}
		\section{二维基础}
			\subsection{点类}
				\begin{lstlisting}
struct Point{
	double x, y;
	Point() {}
	Point(double x, double y):x(x), y(y) {}
	Point operator +(const Point &p)const {
		return Point(x + p.x, y + p.y);
	}
	Point operator -(const Point &p)const {
		return Point(x - p.x, y - p.y);
	}
	Point operator *(const double &p)const {
		return Point(x * p, y * p);
	}
	Point operator /(const double &p)const {
		return Point(x / p, y / p);
	}
	int read() {
		return scanf("%lf%lf", &x, &y);
	}
};
struct Line{
	Point a, b;
	Line() {}
	Line(Point a, Point b):a(a), b(b) {}
};
\end{lstlisting}

			\subsection{凸包}
				\begin{lstlisting}
bool Pair_Comp(const Point &a, const Point &b) {
	if (dcmp(a.x - b.x) < 0) return true;
	if (dcmp(a.x - b.x) > 0) return false;
	return dcmp(a.y - b.y) < 0;
}

int Convex_Hull(int n, Point *P, Point *C) {
	sort(P, P + n, Pair_Comp);
	int top = 0;
	for (int i = 0; i < n; i++) {
		while (top >= 2 && dcmp(det(C[top - 1] - C[top - 2], P[i] - C[top - 2])) <= 0) top--;
		C[top++] = P[i];
	}
	int lasttop = top;
	for (int i = n - 1; i >= 0; i--) {
		while (top > lasttop && dcmp(det(C[top - 1] - C[top - 2], P[i] - C[top - 2])) <= 0) top--;
		C[top++] = P[i];
	}
	return top;
}
\end{lstlisting}

			\subsection{半平面交}
				\begin{lstlisting}
bool isOnLeft(const Point &x, const Line &l) {
	double d = det(x - l.a, l.b - l.a);
	return dcmp(d) <= 0;
}
// 传入一个线段的集合L,传出A,并且返回A的大小
int getIntersectionOfHalfPlane(int n, Line *L, Line *A) {
	Line *q = new Line[n + 1];
	Point *p = new Point[n + 1];
	sort(L, L + n, Polar_Angle_Comp_Line);
	int l = 1, r = 0;
	for (int i = 0; i < n; i++) {
		while (l < r && !isOnLeft(p[r - 1], L[i])) r--;
		while (l < r && !isOnLeft(p[l], L[i])) l++;
		q[++r] = L[i];
		if (l < r && is_Colinear(q[r], q[r - 1])) {
			r--;
			if (isOnLeft(L[i].a, q[r])) q[r] = L[i];
		}
		if (l < r) p[r - 1] = getIntersection(q[r - 1], q[r]);
	}
	while (l < r && !isOnLeft(p[r - 1], q[l])) r--;
	if (r - l + 1 <= 2) return 0;
	int tot = 0;
	for (int i = l; i <= r; i++) A[tot++] = q[i];
	return tot;
}
\end{lstlisting}

			\subsection{最近点对}
				\begin{lstlisting}
bool comparex(const Point &a, const Point &b) {
    return sgn(a.x - b.x) < 0;
}
bool comparey(const Point &a, const Point &b) {
    return sgn(a.y - b.y) < 0;
}
double solve(const std::vector<Point> &point, int left, int right) {
    if (left == right) {
        return INF;
    }
    if (left + 1 == right) {
        return dist(point[left], point[right]);
    }
    int mid = left + right >> 1;
    double result = std::min(solve(left, mid), solve(mid + 1, right));
    std::vector<Point> candidate;
    for (int i = left; i <= right; ++i) {
        if (std::abs(point[i].x - point[mid].x) <= result) {
            candidate.push_back(point[i]);
        }
    }
    std::sort(candidate.begin(), candidate.end(), comparey);
    for (int i = 0; i < (int)candidate.size(); ++i) {
        for (int j = i + 1; j < (int)candidate.size(); ++j) {
            if (std::abs(candidate[i].y - candidate[j].y) >= result) {
                break;
            }
            result = std::min(result, dist(candidate[i], candidate[j]));
        }
    }
    return result;
}
double solve(std::vector<Point> point) {
    std::sort(point.begin(), point.end(), comparex);
    return solve(point, 0, (int)point.size() - 1);
}
\end{lstlisting}

		\section{三维基础}
			\subsection{点类}
				\begin{lstlisting}
int dcmp(const double &x) {
	return fabs(x) < eps ? 0 : (x > 0 ? 1 : -1);
}
struct TPoint{
	double x, y, z;
	TPoint() {}
	TPoint(double x, double y, double z) : x(x), y(y), z(z) {}
	TPoint operator +(const TPoint &p)const {
		return TPoint(x + p.x, y + p.y, z + p.z);
	}
	TPoint operator -(const TPoint &p)const {
		return TPoint(x - p.x, y - p.y, z - p.z);
	}
	TPoint operator *(const double &p)const {
		return TPoint(x * p, y * p, z * p);
	}
	TPoint operator /(const double &p)const {
		return TPoint(x / p, y / p, z / p);
	}
	bool operator <(const TPoint &p)const {
		int dX = dcmp(x - p.x), dY = dcmp(y - p.y), dZ = dcmp(z - p.z);
		return dX < 0 || (dX == 0 && (dY < 0 || (dY == 0 && dZ < 0)));
	}
	bool read() {
		return scanf("%lf%lf%lf", &x, &y, &z) == 3;
	}
};
double sqrdist(const TPoint &a) {
	double ret = 0;
	ret += a.x * a.x;
	ret += a.y * a.y;
	ret += a.z * a.z;
	return ret;
}
double sqrdist(const TPoint &a, const TPoint &b) {
	double ret = 0;
	ret += (a.x - b.x) * (a.x - b.x);
	ret += (a.y - b.y) * (a.y - b.y);
	ret += (a.z - b.z) * (a.z - b.z);
	return ret;
}
double dist(const TPoint &a) {
	return sqrt(sqrdist(a));
}
double dist(const TPoint &a, const TPoint &b) {
	return sqrt(sqrdist(a, b));
}
TPoint det(const TPoint &a, const TPoint &b) {
	TPoint ret;
	ret.x = a.y * b.z - b.y * a.z;
	ret.y = a.z * b.x - b.z * a.x;
	ret.z = a.x * b.y - b.x * a.y;
	return ret;
}
double dot(const TPoint &a, const TPoint &b) {
	double ret = 0;
	ret += a.x * b.x;
	ret += a.y * b.y;
	ret += a.z * b.z;
	return ret;
}
double detdot(const TPoint &a, const TPoint &b, const TPoint &c, const TPoint &d) {
	return dot(det(b - a, c - a), d - a);
}
\end{lstlisting}

			\subsection{凸包}
				\begin{lstlisting}
struct Triangle{
	TPoint a, b, c;
	Triangle() {}
	Triangle(TPoint a, TPoint b, TPoint c) : a(a), b(b), c(c) {}
	double getArea() {
		TPoint ret = det(b - a, c - a);
		return dist(ret) / 2.0;
	}
};
namespace Convex_Hull {
	struct Face{
		int a, b, c;
		bool isOnConvex;
		Face() {}
		Face(int a, int b, int c) : a(a), b(b), c(c) {}
	};
	int nFace, left, right, whe[MAXN][MAXN];
	Face queue[MAXF], tmp[MAXF];
	bool isVisible(const std::vector<TPoint> &p, const Face &f, const TPoint &a) {
		return dcmp(detdot(p[f.a], p[f.b], p[f.c], a)) > 0;
	}
	bool init(std::vector<TPoint> &p) {
		bool check = false;
		for (int i = 1; i < (int)p.size(); i++) {
			if (dcmp(sqrdist(p[0], p[i]))) {
				std::swap(p[1], p[i]);
				check = true;
				break;
			}
		}
		if (!check) return false;
		check = false;
		for (int i = 2; i < (int)p.size(); i++) {
			if (dcmp(sqrdist(det(p[i] - p[0], p[1] - p[0])))) {
				std::swap(p[2], p[i]);
				check = true;
				break;
			}
		}
		if (!check) return false;
		check = false;
		for (int i = 3; i < (int)p.size(); i++) {
			if (dcmp(detdot(p[0], p[1], p[2], p[i]))) {
				std::swap(p[3], p[i]);
				check = true;
				break;
			}
		}
		if (!check) return false;
		for (int i = 0; i < (int)p.size(); i++)
			for (int j = 0; j < (int)p.size(); j++) {
				whe[i][j] = -1;
			}
		return true;
	}
	void pushface(const int &a, const int &b, const int &c) {
		nFace++;
		tmp[nFace] = Face(a, b, c);
		tmp[nFace].isOnConvex = true;
		whe[a][b] = nFace;
		whe[b][c] = nFace;
		whe[c][a] = nFace;
	}
	bool deal(const std::vector<TPoint> &p, const std::pair<int, int> &now, const TPoint &base) {
		int id = whe[now.second][now.first];
		if (!tmp[id].isOnConvex) return true;
		if (isVisible(p, tmp[id], base)) {
			queue[++right] = tmp[id];
			tmp[id].isOnConvex = false;
			return true;
		}
		return false;
	}
	std::vector<Triangle> getConvex(std::vector<TPoint> &p) {
		static std::vector<Triangle> ret;
		ret.clear();
		if (!init(p)) return ret;
		if (!isVisible(p, Face(0, 1, 2), p[3])) pushface(0, 1, 2); else pushface(0, 2, 1);
		if (!isVisible(p, Face(0, 1, 3), p[2])) pushface(0, 1, 3); else pushface(0, 3, 1);
		if (!isVisible(p, Face(0, 2, 3), p[1])) pushface(0, 2, 3); else pushface(0, 3, 2);
		if (!isVisible(p, Face(1, 2, 3), p[0])) pushface(1, 2, 3); else pushface(1, 3, 2);
		for (int a = 4; a < (int)p.size(); a++) {
			TPoint base = p[a];
			for (int i = 1; i <= nFace; i++) {
				if (tmp[i].isOnConvex && isVisible(p, tmp[i], base)) {
					left = 0, right = 0;
					queue[++right] = tmp[i];
					tmp[i].isOnConvex = false;
					while (left < right) {
						Face now = queue[++left];
						if (!deal(p, std::make_pair(now.a, now.b), base)) pushface(now.a, now.b, a);
						if (!deal(p, std::make_pair(now.b, now.c), base)) pushface(now.b, now.c, a);
						if (!deal(p, std::make_pair(now.c, now.a), base)) pushface(now.c, now.a, a);
					}
					break;
				}
			}
		}
		for (int i = 1; i <= nFace; i++) {
			Face now = tmp[i];
			if (now.isOnConvex) {
				ret.push_back(Triangle(p[now.a], p[now.b], p[now.c]));
			}
		}
		return ret;
	}
};
int n;
std::vector<TPoint> p;
std::vector<Triangle> answer;
int main() {
	scanf("%d", &n);
	for (int i = 1; i <= n; i++) {
		TPoint a;
		a.read();
		p.push_back(a);
	}
	answer = Convex_Hull::getConvex(p);
	double areaCounter = 0.0;
	for (int i = 0; i < (int)answer.size(); i++) {
		areaCounter += answer[i].getArea();
	}
	printf("%.3f\n", areaCounter);
	return 0;
}
\end{lstlisting}

			\subsection{绕轴旋转}
				使用方法及注意事项:逆时针绕轴$AB$旋转$\theta$角
				\begin{lstlisting}
Matrix getTrans(const double &a, const double &b, const double &c) {
    Matrix ret;
    ret.a[0][0] = 1; ret.a[0][1] = 0; ret.a[0][2] = 0; ret.a[0][3] = 0;
    ret.a[1][0] = 0; ret.a[1][1] = 1; ret.a[1][2] = 0; ret.a[1][3] = 0;
    ret.a[2][0] = 0; ret.a[2][1] = 0; ret.a[2][2] = 1; ret.a[2][3] = 0;
    ret.a[3][0] = a; ret.a[3][1] = b; ret.a[3][2] = c; ret.a[3][3] = 1;
    return ret;
}
Matrix getRotate(const double &a, const double &b, const double &c, const double &theta) {
    Matrix ret;
    ret.a[0][0] = a * a * (1 - cos(theta)) + cos(theta);
    ret.a[0][1] = a * b * (1 - cos(theta)) + c * sin(theta);
    ret.a[0][2] = a * c * (1 - cos(theta)) - b * sin(theta);
    ret.a[0][3] = 0;
    ret.a[1][0] = b * a * (1 - cos(theta)) - c * sin(theta);
    ret.a[1][1] = b * b * (1 - cos(theta)) + cos(theta);
    ret.a[1][2] = b * c * (1 - cos(theta)) + a * sin(theta);
    ret.a[1][3] = 0;
    ret.a[2][0] = c * a * (1 - cos(theta)) + b * sin(theta);
    ret.a[2][1] = c * b * (1 - cos(theta)) - a * sin(theta);
    ret.a[2][2] = c * c * (1 - cos(theta)) + cos(theta);
    ret.a[2][3] = 0;
    ret.a[3][0] = 0;
    ret.a[3][1] = 0;
    ret.a[3][2] = 0;
    ret.a[3][3] = 1;
    return ret;
}
Matrix getRotate(const double &ax, const double &ay, const double &az, const double &bx, const double &by, const double &bz, const double &theta) {
    double l = dist(Point(0, 0, 0), Point(bx, by, bz));
    Matrix ret = getTrans(-ax, -ay, -az);
    ret = ret * getRotate(bx / l, by / l, bz / l, theta);
    ret = ret * getTrans(ax, ay, az);
    return ret;
}
\end{lstlisting}

		\section{多边形}
			\subsection{判断点在多边形内部}
				\begin{lstlisting}
bool point_on_line(const Point &p, const Point &a, const Point &b) {
    return sgn(det(p, a, b)) == 0 && sgn(dot(p, a, b)) <= 0;
}

bool point_in_polygon(const Point &p, const std::vector<Point> &polygon) {
    int counter = 0;
    for (int i = 0; i < (int)polygon.size(); ++i) {
        Point a = polygon[i], b = polygon[(i + 1) % (int)polygon.size()];
        if (point_on_line(p, a, b)) {
            //    Point on the boundary are excluded.
            return false;
        }
        int x = sgn(det(a, p, b));
        int y = sgn(a.y - p.y);
        int z = sgn(b.y - p.y);
        counter += (x > 0 && y <= 0 && z > 0);
        counter -= (x < 0 && z <= 0 && y > 0);
    }
    return counter;
}
\end{lstlisting}

			\subsection{多边形内整点计数}
				\begin{lstlisting}
int getInside(int n, Point *P) {  // 求多边形P内有多少个整数点
	int OnEdge = n;
	double area = getArea(n, P);
	for (int i = 0; i < n - 1; i++) {
		Point now = P[i + 1] - P[i];
		int y = (int)now.y, x = (int)now.x;
		OnEdge += abs(gcd(x, y)) - 1;
	}
	Point now = P[0] - P[n - 1];
	int y = (int)now.y, x = (int)now.x;
	OnEdge += abs(gcd(x, y)) - 1;
	double ret = area - (double)OnEdge / 2 + 1;
	return (int)ret;
}
\end{lstlisting}

			%\subsection{旋转卡壳}
			%\subsection{动态凸包}
			%\subsection{点到凸包的切线}
			%\subsection{直线与凸包的交点}
			%\subsection{凸多边形的交集}
			%\subsection{凸多边形内的最大圆}
		\section{圆}
			%\subsubsection{圆类}
			%\subsection{圆的交集}
			\subsection{最小覆盖圆}
				\begin{lstlisting}
Point getmid(Point a,Point b) {
	return Point((a.x + b.x) / 2, (a.y + b.y) / 2);
}
Point getcross(Point a, Point vA, Point b, Point vB) {
	Point u = a - b;
	double t = det(vB, u) / det(vA, vB);
	return a + vA * t;
}
Point getcir(Point a,Point b,Point c) {
	Point midA = getmid(a,b), vA = Point(-(b - a).y, (b - a).x);
	Point midB = getmid(b,c), vB = Point(-(c - b).y, (c - b).x);
	return getcross(midA, vA, midB, vB);
}
double mincir(Point *p,int n) {
	std::random_shuffle(p + 1, p + n + 1);
	Point O = p[1];
	double r = 0;
	for (int i = 2; i <= n; i++) {
		if (dist(O, p[i]) <= r) continue;
		O = p[i]; r = 0;
		for (int j = 1; j < i; j++) {
			if (dist(O, p[j]) <= r) continue;
			O = getmid(p[i], p[j]); r = dist(O,p[i]);
			for (int k = 1; k < j; k++) {
				if (dist(O,p[k]) <= r) continue;
				O = getcir(p[i], p[j], p[k]);
				r = dist(O,p[i]);
			}
		}
	}
	return r;
}
\end{lstlisting}

			\subsection{最小覆盖球}
				\input{Source/Computational-Geometry/Minimum-Coverage-Ball.tex}
			%\subsubsection{判断圆存在交集}
			\subsection{多边形与圆的交面积}
				\begin{lstlisting}
// 求扇形面积
double getSectorArea(const Point &a, const Point &b, const double &r) {
	double c = (2.0 * r * r - sqrdist(a, b)) / (2.0 * r * r);
	double alpha = acos(c);
	return r * r * alpha / 2.0;
}
// 求二次方程ax^2 + bx + c = 0的解
std::pair<double, double> getSolution(const double &a, const double &b, const double &c) {
	double delta = b * b - 4.0 * a * c;
	if (dcmp(delta) < 0) return std::make_pair(0, 0);
	else return std::make_pair((-b - sqrt(delta)) / (2.0 * a), (-b + sqrt(delta)) / (2.0 * a));
}
// 直线与圆的交点
std::pair<Point, Point> getIntersection(const Point &a, const Point &b, const double &r) {
	Point d = b - a;
	double A = dot(d, d);
	double B = 2.0 * dot(d, a);
	double C = dot(a, a) - r * r;
	std::pair<double, double> s = getSolution(A, B, C);
	return std::make_pair(a + d * s.first, a + d * s.second);
}
// 原点到线段AB的距离
double getPointDist(const Point &a, const Point &b) {
	Point d = b - a;
	int sA = dcmp(dot(a, d)), sB = dcmp(dot(b, d));
	if (sA * sB <= 0) return det(a, b) / dist(a, b);
	else return std::min(dist(a), dist(b));
}
// a和b和原点组成的三角形与半径为r的圆的交的面积
double getArea(const Point &a, const Point &b, const double &r) {
	double dA = dot(a, a), dB = dot(b, b), dC = getPointDist(a, b), ans = 0.0;
	if (dcmp(dA - r * r) <= 0 && dcmp(dB - r * r) <= 0) return det(a, b) / 2.0;
	Point tA = a / dist(a) * r;
	Point tB = b / dist(b) * r;
	if (dcmp(dC - r) > 0) return getSectorArea(tA, tB, r);
	std::pair<Point, Point> ret = getIntersection(a, b, r);
	if (dcmp(dA - r * r) > 0 && dcmp(dB - r * r) > 0) {
		ans += getSectorArea(tA, ret.first, r);
		ans += det(ret.first, ret.second) / 2.0;
		ans += getSectorArea(ret.second, tB, r);
		return ans;
	}
	if (dcmp(dA - r * r) > 0) return det(ret.first, b) / 2.0 + getSectorArea(tA, ret.first, r);
	else return det(a, ret.second) / 2.0 + getSectorArea(ret.second, tB, r);
}
// 求圆与多边形的交的主过程
double getArea(int n, Point *p, const Point &c, const double r)  {
	double ret = 0.0;
	for (int i = 0; i < n; i++) {
		int sgn = dcmp(det(p[i] - c, p[(i + 1) % n] - c));
		if (sgn > 0) ret += getArea(p[i] - c, p[(i + 1) % n] - c, r);
		else ret -= getArea(p[(i + 1) % n] - c, p[i] - c, r);
	}
	return fabs(ret);
}
\end{lstlisting}

		%\subsection{三角形}
			%\subsubsection{三角形的内心}
			%\subsubsection{三角形的外心}
			%\subsubsection{三角形的垂心}
		%\subsection{黑暗科技}
			%\subsubsection{平面图形的转动惯量}
			%\subsubsection{平面区域处理}
			%\subsubsection{Vonoroi图}
	\chapter{其它}
		\section{STL使用方法}
			\subsection{nth\_element}
	用法:nth\_element(a + 1, a + id, a + n + 1); \par
	作用:将排名为$id$的元素放在第$id$个位置。
\subsection{next\_permutation}
	用法:next\_permutation(a + 1, a + n + 1); \par
	作用:以a中从小到大排序后为第一个排列,求得当期数组a中的下一个排列,返回值为当期排列是否为最后一个排列。

		\section{博弈论相关}
			\subsection{巴什博奕}
	\begin{enumerate}
		\item 
			只有一堆n个物品,两个人轮流从这堆物品中取物,规定每次至少取一个,最多取m个。最后取光者得胜。
		\item
			显然,如果$n=m+1$,那么由于一次最多只能取$m$个,所以,无论先取者拿走多少个,
			后取者都能够一次拿走剩余的物品,后者取胜。因此我们发现了如何取胜的法则:如果
			$n=(m+1)r+s$,(r为任意自然数,$s \leq m$),那么先取者要拿走$s$个物品,
			如果后取者拿走$k(k \leq m)$个,那么先取者再拿走$m+1-k$个,结果剩下$(m+1)(r-1)$
			个,以后保持这样的取法,那么先取者肯定获胜。总之,要保持给对手留下$(m+1)$的倍数,
			就能最后获胜。
	\end{enumerate}
\subsection{威佐夫博弈}
	\begin{enumerate}
		\item 
			有两堆各若干个物品,两个人轮流从某一堆或同时从两堆中取同样多的物品,规定每次至少取
			一个,多者不限,最后取光者得胜。
		\item
			判断一个局势$(a, b)$为奇异局势(必败态)的方法:
			$$a_k =[k (1+\sqrt{5})/2],b_k= a_k + k$$
	\end{enumerate}
\subsection{阶梯博奕}
	\begin{enumerate}
		\item
			博弈在一列阶梯上进行,每个阶梯上放着自然数个点,两个人进行阶梯博弈,
			每一步则是将一个阶梯上的若干个点(至少一个)移到前面去,最后没有点
			可以移动的人输。
		\item
			解决方法:把所有奇数阶梯看成N堆石子,做NIM。(把石子从奇数堆移动到偶数
			堆可以理解为拿走石子,就相当于几个奇数堆的石子在做Nim)
	\end{enumerate}
\subsection{图上删边游戏}
	\subsubsection{链的删边游戏}
		\begin{enumerate}
			\item
				游戏规则:对于一条链,其中一个端点是根,两人轮流删边,脱离根的部分也算被删去,最后没边可删的人输。
			\item
				做法:$sg[i] = n - dist(i) - 1$(其中$n$表示总点数,$dist(i)$表示离根的距离)
		\end{enumerate}
	\subsubsection{树的删边游戏}
		\begin{enumerate}
			\item
				游戏规则:对于一棵有根树,两人轮流删边,脱离根的部分也算被删去,没边可删的人输。
			\item
				做法:叶子结点的$sg=0$,其他节点的$sg$等于儿子结点的$sg+1$的异或和。
		\end{enumerate}
	\subsubsection{局部连通图的删边游戏}
		\begin{enumerate}
			\item
				游戏规则:在一个局部连通图上,两人轮流删边,脱离根的部分也算被删去,没边可删的人输。
				局部连通图的构图规则是,在一棵基础树上加边得到,所有形成的环保证不共用边,且只与基础树有一个公共点。
			\item
				做法:去掉所有的偶环,将所有的奇环变为长度为1的链,然后做树的删边游戏。
		\end{enumerate}

		\section{Java Reference}
			\lstset{
	language=JAVA,
	tabsize=2,
	numbers=left,
	extendedchars=false,
	showstringspaces=false,
	xleftmargin=0em,
	xrightmargin=0em,
	frame=trbl,
	numberstyle=\scriptsize\Courier,
    basicstyle=\scriptsize\Courier
}
\begin{lstlisting}
import java.io.*;
import java.util.*;
import java.math.*;
public class Main {
	static int get(char c) {
		if (c <= '9')
			return c - '0';
		else if (c <= 'Z')
			return c - 'A' + 10;
		else
			return c - 'a' + 36;
	}
	static char get(int x) {
		if (x <= 9)
			return (char)(x + '0');
		else if (x <= 35)
			return (char)(x - 10 + 'A');
		else
			return (char)(x - 36 + 'a');
	}
	static BigInteger get(String s, BigInteger x) {
		BigInteger ans = BigInteger.valueOf(0), now = BigInteger.valueOf(1);
		for (int i = s.length() - 1; i >= 0; i--) {
			ans = ans.add(now.multiply(BigInteger.valueOf(get(s.charAt(i)))));
			now = now.multiply(x);
		}
		return ans;
	}
	public static void main(String [] args) {
		Scanner cin = new Scanner(new BufferedInputStream(System.in));
		for (; ; ) {
			BigInteger x = cin.nextBigInteger();
			if (x.compareTo(BigInteger.valueOf(0)) == 0)
				break;
			String s = cin.next(), t = cin.next(), r = "";
			BigInteger ans = get(s, x).mod(get(t, x));
			if (ans.compareTo(BigInteger.valueOf(0)) == 0)
				r = "0";
			for (; ans.compareTo(BigInteger.valueOf(0)) > 0;) {
				r = get(ans.mod(x).intValue()) + r;
				ans = ans.divide(x);
			}
			System.out.println(r);
		}
	}
}
// Arrays
int a[];
.fill(a[, int fromIndex, int toIndex],val); | .sort(a[, int fromIndex, int toIndex])
// String
String s;
.charAt(int i); | compareTo(String) | compareToIgnoreCase () | contains(String) |
length () | substring(int l, int len)
// BigInteger
.abs() | .add() | bitLength () | subtract () | divide () | remainder () | divideAndRemainder () | modPow(b, c) |
pow(int) | multiply () | compareTo () |
gcd() | intValue () | longValue () | isProbablePrime(int c) (1 - 1/2^c) |
nextProbablePrime () | shiftLeft(int) | valueOf ()
// BigDecimal
.ROUND_CEILING | ROUND_DOWN_FLOOR | ROUND_HALF_DOWN | ROUND_HALF_EVEN | ROUND_HALF_UP | ROUND_UP
.divide(BigDecimal b, int scale , int round_mode) | doubleValue () | movePointLeft(int) | pow(int) |
setScale(int scale , int round_mode) | stripTrailingZeros ()
// StringBuilder
StringBuilder sb = new StringBuilder ();
sb.append(elem) | out.println(sb)
\end{lstlisting}

	\chapter{数学公式}
		\section{常用数学公式}
	\subsection{求和公式}
		\begin{enumerate}
			\item $\sum_{k=1}^{n}(2k-1)^2 = \frac{n(4n^2-1)}{3}	$
			\item $\sum_{k=1}^{n}k^3 = [\frac{n(n+1)}{2}]^2	$
			\item $\sum_{k=1}^{n}(2k-1)^3 = n^2(2n^2-1)	$
			\item $\sum_{k=1}^{n}k^4 = \frac{n(n+1)(2n+1)(3n^2+3n-1)}{30}  $
			\item $\sum_{k=1}^{n}k^5 = \frac{n^2(n+1)^2(2n^2+2n-1)}{12}	$
			\item $\sum_{k=1}^{n}k(k+1) = \frac{n(n+1)(n+2)}{3}	$
			\item $\sum_{k=1}^{n}k(k+1)(k+2) = \frac{n(n+1)(n+2)(n+3)}{4} $
			\item $\sum_{k=1}^{n}k(k+1)(k+2)(k+3) = \frac{n(n+1)(n+2)(n+3)(n+4)}{5} $
		\end{enumerate}
	\subsection{斐波那契数列}
		\begin{enumerate}
			\item $fib_0=0, fib_1=1, fib_n=fib_{n-1}+fib_{n-2}$
			\item $fib_{n+2} \cdot fib_n-fib_{n+1}^2=(-1)^{n+1}$
			\item $fib_{-n}=(-1)^{n-1}fib_n$
			\item $fib_{n+k}=fib_k \cdot fib_{n+1}+fib_{k-1} \cdot fib_n$
			\item $gcd(fib_m, fib_n)=fib_{gcd(m, n)}$
			\item $fib_m|fib_n^2\Leftrightarrow nfib_n|m$
		\end{enumerate}
	\subsection{错排公式}
		\begin{enumerate}
			\item $D_n = (n-1)(D_{n-2}-D_{n-1})$
			\item $D_n = n! \cdot (1-\frac{1}{1!}+\frac{1}{2!}-\frac{1}{3!}+\ldots+\frac{(-1)^n}{n!})$
		\end{enumerate}
	\subsection{莫比乌斯函数}
		$$\mu(n) = \begin{cases}
			1 & \text{若}n=1\\
			(-1)^k & \text{若}n\text{无平方数因子,且}n = p_1p_2\dots p_k\\
			0 & \text{若}n\text{有大于}1\text{的平方数因数}
		\end{cases}$$
		$$\sum_{d|n}{\mu(d)} = \begin{cases}
			1 & \text{若}n=1\\
			0 & \text{其他情况}
		\end{cases}$$
		$$g(n) = \sum_{d|n}{f(d)} \Leftrightarrow f(n) = \sum_{d|n}{\mu(d)g(\frac{n}{d})}$$
		$$g(x) = \sum_{n=1}^{[x]}f(\frac{x}{n}) \Leftrightarrow f(x) = \sum_{n=1}^{[x]}{\mu(n)g(\frac{x}{n})}$$
	\subsection{Burnside引理}
		设$G$是一个有限群,作用在集合$X$上。对每个$g$属于$G$,令$X^g$表示$X$中在$g$作用下的不动元素,轨道数(记作$|X/G|$)由如下公式给出:
			$$|X/G| = \frac{1}{|G|}\sum_{g \in G}|X^g|.\,$$
	\subsection{五边形数定理}
		设$p(n)$是$n$的拆分数,有$$p(n) = \sum_{k \in \mathbb{Z} \setminus \{0\}} (-1)^{k - 1} p\left(n - \frac{k(3k - 1)}{2}\right)$$
	\subsection{树的计数}
		\begin{enumerate}
			\item 有根树计数:$n+1$个结点的有根树的个数为
				$$a_{n+1} = \frac{\sum_{j=1}^{n}{j \cdot a_j \cdot{S_{n, j}}}}{n}$$
			其中,
				$$S_{n, j} = \sum_{i=1}^{n/j}{a_{n+1-ij}} = S_{n-j, j} + a_{n+1-j}$$
			\item 无根树计数:当$n$为奇数时,$n$个结点的无根树的个数为
				$$a_n-\sum_{i=1}^{n/2}{a_ia_{n-i}}$$
			当$n$为偶数时,$n$个结点的无根树的个数为
				$$a_n-\sum_{i=1}^{n/2}{a_ia_{n-i}}+\frac{1}{2}a_{\frac{n}{2}}(a_{\frac{n}{2}}+1)$$
			\item $n$个结点的完全图的生成树个数为
				$$n^{n-2}$$
			\item 矩阵-树定理:
			图$G$由$n$个结点构成,设$\bm{A}[G]$为图$G$的邻接矩阵、$\bm{D}[G]$为图$G$的度数矩阵,
			则图$G$的不同生成树的个数为$\bm{C}[G] = \bm{D}[G] - \bm{A}[G]$的任意一个$n-1$阶主子式的行列式值。
		\end{enumerate}
	\subsection{欧拉公式}
		平面图的顶点个数、边数和面的个数有如下关系:
			$$V - E + F = C+ 1$$
		\indent 其中,$V$是顶点的数目,$E$是边的数目,$F$是面的数目,$C$是组成图形的连通部分的数目。当图是单连通图的时候,公式简化为:
			$$V - E + F = 2$$
	\subsection{皮克定理}
		给定顶点坐标均是整点(或正方形格点)的简单多边形,其面积$A$和内部格点数目$i$、边上格点数目$b$的关系:
			$$A = i + \frac{b}{2} - 1$$
	\subsection{牛顿恒等式}
		设$$\prod_{i = 1}^n{(x - x_i)} = a_n + a_{n - 1} x + \dots + a_1 x^{n - 1} + a_0 x^n$$
		$$p_k = \sum_{i = 1}^n{x_i^k}$$
		则$$a_0 p_k + a_1 p_{k - 1} + \cdots + a_{k - 1} p_1 + k a_k = 0$$\par
		特别地,对于$$|\bm{A} - \lambda \bm{E}| = (-1)^n(a_n + a_{n - 1} \lambda + \cdots + a_1 \lambda^{n - 1} + a_0 \lambda^n)$$
		有$$p_k = Tr(\bm{A}^k)$$
	%\section{数论公式}
\section{平面几何公式}
	\subsection{三角形}
		\begin{enumerate}
			\item 半周长
				$$p=\frac{a+b+c}{2}$$
			\item 面积
				$$S=\frac{a \cdot H_a}{2}=\frac{ab \cdot sinC}{2}=\sqrt{p(p-a)(p-b)(p-c)}$$
			\item 中线
				$$M_a=\frac{\sqrt{2(b^2+c^2)-a^2}}{2}=\frac{\sqrt{b^2+c^2+2bc \cdot cosA}}{2}$$
			\item 角平分线 
				$$T_a=\frac{\sqrt{bc \cdot [(b+c)^2-a^2]}}{b+c}=\frac{2bc}{b+c}cos\frac{A}{2}$$
			\item 高线
				$$H_a=bsinC=csinB=\sqrt{b^2-(\frac{a^2+b^2-c^2}{2a})^2}$$
			\item 内切圆半径
				\begin{align*}
					r&=\frac{S}{p}=\frac{arcsin\frac{B}{2} \cdot sin\frac{C}{2}}{sin\frac{B+C}{2}}=4R \cdot sin\frac{A}{2}sin\frac{B}{2}sin\frac{C}{2}\\
					&=\sqrt{\frac{(p-a)(p-b)(p-c)}{p}}=p \cdot tan\frac{A}{2}tan\frac{B}{2}tan\frac{C}{2}
				\end{align*}
			\item 外接圆半径
				$$R=\frac{abc}{4S}=\frac{a}{2sinA}=\frac{b}{2sinB}=\frac{c}{2sinC}$$
		\end{enumerate}
	\subsection{四边形}
		$D_1, D_2$为对角线,$M$对角线中点连线,$A$为对角线夹角,$p$为半周长
		\begin{enumerate}
			\item $a^2+b^2+c^2+d^2=D_1^2+D_2^2+4M^2$
			\item $S=\frac{1}{2}D_1D_2sinA$
			\item 对于圆内接四边形
				$$ac+bd=D_1D_2$$
			\item 对于圆内接四边形
				$$S=\sqrt{(p-a)(p-b)(p-c)(p-d)}$$
		\end{enumerate}
	\subsection{正$n$边形}
		$R$为外接圆半径,$r$为内切圆半径
		\begin{enumerate}
			\item 中心角
				$$A=\frac{2\pi}{n}$$
			\item 内角
				$$C=\frac{n-2}{n}\pi$$
			\item 边长
				$$a=2\sqrt{R^2-r^2}=2R \cdot sin\frac{A}{2}=2r \cdot tan\frac{A}{2}$$
			\item 面积
				$$S=\frac{nar}{2}=nr^2 \cdot tan\frac{A}{2}=\frac{nR^2}{2} \cdot sinA=\frac{na^2}{4 \cdot tan\frac{A}{2}}$$
		\end{enumerate}
	\subsection{圆}
		\begin{enumerate}
			\item 弧长
				$$l=rA$$
			\item 弦长
				$$a=2\sqrt{2hr-h^2}=2r\cdot sin\frac{A}{2}$$
			\item 弓形高
				$$h=r-\sqrt{r^2-\frac{a^2}{4}}=r(1-cos\frac{A}{2})=\frac{1}{2} \cdot arctan\frac{A}{4}$$
			\item 扇形面积
				$$S_1=\frac{rl}{2}=\frac{r^2A}{2}$$
			\item 弓形面积
				$$S_2=\frac{rl-a(r-h)}{2}=\frac{r^2}{2}(A-sinA)$$
		\end{enumerate}
	\subsection{棱柱}
		\begin{enumerate}
			\item 体积
				$$V=Ah$$
				$A$为底面积,$h$为高
			\item 侧面积
				$$S=lp$$
				$l$为棱长,$p$为直截面周长
			\item 全面积
				$$T=S+2A$$
		\end{enumerate}
	\subsection{棱锥}
		\begin{enumerate}
			\item 体积
				$$V=Ah$$
				$A$为底面积,$h$为高
			\item 正棱锥侧面积
				$$S=lp$$
				$l$为棱长,$p$为直截面周长
			\item 正棱锥全面积
				$$T=S+2A$$
		\end{enumerate}
	\subsection{棱台}
		\begin{enumerate}
			\item 体积
				$$V=(A_1+A_2+\sqrt{A_1A_2}) \cdot \frac{h}{3}$$
				$A_1,A_2$为上下底面积,$h$为高
			\item 正棱台侧面积
				$$S=\frac{p_1+p_2}{2}l$$
				$p_1,p_2$为上下底面周长,$l$为斜高
			\item 正棱台全面积
				$$T=S+A_1+A_2$$
		\end{enumerate}
	\subsection{圆柱}
		\begin{enumerate}
			\item 侧面积
				$$S=2\pi rh$$
			\item 全面积
				$$T=2\pi r(h+r)$$
			\item 体积
				$$V=\pi r^2h$$
		\end{enumerate}
	\subsection{圆锥}
		\begin{enumerate}
			\item 母线
				$$l=\sqrt{h^2+r^2}$$
			\item 侧面积
				$$S=\pi rl$$
			\item 全面积
				$$T=\pi r(l+r)$$
			\item 体积
				$$V=\frac{\pi}{3} r^2h$$
		\end{enumerate}
	\subsection{圆台}
		\begin{enumerate}
			\item 母线
				$$l=\sqrt{h^2+(r_1-r_2)^2}$$
			\item 侧面积
				$$S=\pi(r_1+r_2)l$$
			\item 全面积
				$$T=\pi r_1(l+r_1)+\pi r_2(l+r_2)$$
			\item 体积
				$$V=\frac{\pi}{3}(r_1^2+r_2^2+r_1r_2)h$$
		\end{enumerate}
	\subsection{球}
		\begin{enumerate}
			\item 全面积
				$$T=4\pi r^2$$
			\item 体积
				$$V=\frac{4}{3}\pi r^3$$
		\end{enumerate}
	\subsection{球台}
		\begin{enumerate}
			\item 侧面积
				$$S=2\pi rh$$
			\item 全面积
				$$T=\pi(2rh+r_1^2+r_2^2)$$
			\item 体积
				$$V=\frac{\pi h[3(r_1^2+r_2^2)+h^2]}{6}$$
		\end{enumerate}
	\subsection{球扇形}
		\begin{enumerate}
			\item 全面积
				$$T=\pi r(2h+r_0)$$
				$h$为球冠高,$r_0$为球冠底面半径
			\item 体积
				$$V=\frac{2}{3}\pi r^2h$$
		\end{enumerate}
\section{立体几何公式}
	\subsection{球面三角公式}
		设$a, b, c$是边长,$A, B, C$是所对的二面角,
		有余弦定理$$cos a = cos b \cdot cos c + sin b \cdot sin c \cdot cos A$$
		正弦定理$$\frac{sin A}{sin a} = \frac{sin B}{sin b} = \frac{sin C}{sin c}$$
		三角形面积是$A + B + C - \pi$
	\subsection{四面体体积公式}
		$U, V, W, u, v, w$是四面体的$6$条棱,$U, V, W$构成三角形,$(U, u), (V, v), (W, w)$互为对棱,
		则$$V = \frac{\sqrt{(s - 2a)(s - 2b)(s - 2c)(s - 2d)}}{192 uvw}$$
		其中$$\left\{\begin{array}{lll}
				a & = & \sqrt{xYZ}, \\
				b & = & \sqrt{yZX}, \\
				c & = & \sqrt{zXY}, \\
				d & = & \sqrt{xyz}, \\
				s & = & a + b + c + d, \\ 
				X & = & (w - U + v)(U + v + w), \\
				x & = & (U - v + w)(v - w + U), \\
				Y & = & (u - V + w)(V + w + u), \\
				y & = & (V - w + u)(w - u + V), \\
				Z & = & (v - W + u)(W + u + v), \\
				z & = & (W - u + v)(u - v + W)
			\end{array}\right.$$

	\end{spacing}
\end{document}
