\section{图论}
\subsection{2-SAT}
\inputminted{cpp}{\source/graph-theory/sat-lexicographically.cpp}
\subsection{2-SAT(tarjan)}
\inputminted{cpp}{\source/graph-theory/2-sat-tarjan.cpp}
\subsection{KM}
\inputminted{cpp}{\source/graph-theory/KM.cpp}
\subsection{点双连通分量}
\inputminted{cpp}{\source/graph-theory/biconnected-graph-vertex.cpp}
\subsection{边双连通分量}
\inputminted{cpp}{\source/graph-theory/biconnected-graph-edge.cpp}
\subsection{最小树形图}
\inputminted{cpp}{\source/graph-theory/optimum-branching.cpp}
\subsection{带花树}
\inputminted{cpp}{\source/graph-theory/blossom-algorithm.cpp}
\subsection{支配树}
\inputminted{cpp}{\source/graph-theory/dominator-tree.cpp}
\subsection{无向图最小割}
\inputminted{cpp}{\source/graph-theory/stoer-wagner-algorithm.cpp}
\subsection{最大团搜索}
\inputminted{cpp}{\source/graph-theory/maxclique.cpp}
\subsection{弦图判定}
\inputminted{cpp}{\source/graph-thery/judge.cpp}
\subsection{斯坦纳树}
\inputminted{cpp}{\source/graph-theory/Steiner-Tree.cpp}
\subsection{虚树}
\inputminted{cpp}{\source/graph-theory/mirage-tree.cpp}
\subsection{点分治}
\inputminted{cpp}{\source/graph-theory/vertex-partition.cpp}
\subsection{最小割最大流}
\inputminted{cpp}{\source/graph-theory/dinic.cpp}
\subsection{最小费用流}
\inputminted{cpp}{\source/graph-theory/mincost-maxflow.cpp}
\subsection{zkw费用流}
\inputminted{cpp}{\source/graph-theory/zkw-cost-flow.cpp}
\subsection{最小割树}
\inputminted{cpp}{\source/graph-theory/GH-tree.cpp}
\subsection{上下界网络流建图}
$B(u,v)$表示边$(u,v)$流量的下界,$C(u,v)$表示边$(u,v)$流量的上界,$F(u,v)$表示边$(u,v)$的流量。
设$G(u,v) = F(u,v) - B(u,v)$,显然有
$$0 \leq G(u,v) \leq C(u,v)-B(u,v)$$
\subsubsection{无源汇的上下界可行流}
建立超级源点$S^*$和超级汇点$T^*$,对于原图每条边$(u,v)$在新网络中连如下三条边:$S^* \rightarrow v$,容量为$B(u,v)$;$u \rightarrow T^*$,容量为$B(u,v)$;$u \rightarrow v$,容量为$C(u,v) - B(u,v)$。最后求新网络的最大流,判断从超级源点$S^*$出发的边是否都满流即可,边$(u,v)$的最终解中的实际流量为$G(u,v)+B(u,v)$。
\subsubsection{有源汇的上下界可行流}
从汇点$T$到源点$S$连一条上界为$\infty$,下界为$0$的边。按照\textbf{无源汇的上下界可行流}一样做即可,流量即为$T \rightarrow S$边上的流量。
\subsubsection{有源汇的上下界最大流}
\begin{enumerate}
	\item 在\textbf{有源汇的上下界可行流}中,从汇点$T$到源点$S$的边改为连一条上界为$\infty$,下届为$x$的边。$x$满足二分性质,找到最大的$x$使得新网络存在\textbf{无源汇的上下界可行流}即为原图的最大流。
	\item 从汇点$T$到源点$S$连一条上界为$\infty$,下界为$0$的边,变成无源汇的网络。按照\textbf{无源汇的上下界可行流}的方法,建立超级源点$S^*$和超级汇点$T^*$,求一遍$S^* \rightarrow T^*$的最大流,再将从汇点$T$到源点$S$的这条边拆掉,求一次$S \rightarrow T$的最大流即可。
\end{enumerate}
\subsubsection{有源汇的上下界最小流}
\begin{enumerate}
	\item 在\textbf{有源汇的上下界可行流}中,从汇点$T$到源点$S$的边改为连一条上界为$x$,下界为$0$的边。$x$满足二分性质,找到最小的$x$使得新网络存在\textbf{无源汇的上下界可行流}即为原图的最小流。
	\item 按照\textbf{无源汇的上下界可行流}的方法,建立超级源点$S^*$与超级汇点$T^*$,求一遍$S^* \rightarrow T^*$的最大流,但是注意这一次不加上汇点$T$到源点$S$的这条边,即不使之改为无源汇的网络去求解。求完后,再加上那条汇点$T$到源点$S$上界$\infty$的边。因为这条边下界为$0$,所以$S^*$,$T^*$无影响,再直接求一次$S^* \rightarrow T^*$的最大流。若超级源点$S^*$出发的边全部满流,则$T \rightarrow S$边上的流量即为原图的最小流,否则无解。
\end{enumerate}
